\chapter{Grafico probabile}
\section{Funzioni razionali intere}
Risolvi i seguenti esercizi
\tcbstartrecording
%\begin{exercise}

%	\tcblower
%	\begin{itemize}
%	\litem {Dominio}
%	
%	\litem{Positività}
%	
%	Otteniamo il grafico:
%	
%	\litem{Asintoti}
%	\begin{enumerate}
%		\item Asintoti verticali
%		
%		Utilizzando il grafico precedente
%		
%		\item Asintoto orizzontale 
%		
%		\item Asintoto obliquo
%		
%	\end{enumerate}
%	\litem{Intersezioni}
%	\begin{enumerate}
%		\item Asse $x$
%		\item asse $y$
%		
%		\item Asintoti
%		
%	\end{enumerate}
%	\litem{Grafico probabile}
%\end{itemize}
%\end{exercise}
%\begin{exercise}[no solution]
%	It holds:
%	\begin{equation*}
%	\frac{d}{dx}\left(\ln|x|\right) = \frac{1}{x}.
%	\end{equation*}
%\end{exercise}
\begin{exercise}[no solution]
	Disegnare il grafico probabile di questa funzione $f(x)=(x^2-1)(x^2-4)$
\end{exercise}
\begin{exercise}
	Disegnare il grafico probabile di questa funzione $f(x)=(x^2-1)(x-4)$
\tcblower
	Disegnare il grafico probabile di questa funzione $f(x)=(x^2-1)(x-4)$
	\begin{itemize}
	\litem {Dominio}
	La funzione è intera quindi è definita su tutto $\R$
	\litem{Positività}
	Studiamo il segno di $f(x)=(x^2-1)(x-4)$
	\begin{align*}
(x^2-1)(x-4)\geq&0\\
\intertext{Risolviamo le due disequazioni}
x-4\geq&0\\
x\geq&4\\
x^2-1\geq&0\\
x^2-1=&0\\
x_{1,2}=&\dfrac{\pm\sqrt{4}}{2}
=\begin{cases}
x_1=+1\\
x_2=-1
\end{cases}\\
	\end{align*}
Otteniamo il grafico:
\begin{center}
	\includestandalone{quinto/grafici/disgraficoprob11}
\end{center}
Dal diagramma vediamo che la funzione è positiva per valori di $x$ compresi tra meno uno e più uno e per valori di $x$ maggiori di quattro $-1\leq x \leq 1$ o $x\geq 4$
	\litem{Asintoti}
	\begin{enumerate}
		\item Asintoti verticali
		
		Visto il dominio non esistono asintoti verticali.
		\item Asintoto orizzontale 
		Calcoliamo il limite
		\begin{align*}
		\lim_{x\to +\infty}(x^2-1)(x-4)=&\\
		\lim_{x\to +\infty}(x^3-4x^2-x+4)=&\\
		\lim_{x\to +\infty}x^3(1-\dfrac{4x^2}{x^3}-\dfrac{x}{x^3}+\dfrac{4}{x^3})=&\infty\cdot 1=\infty\\
		\end{align*}
		Quindi non esiste asintoto orizzontale
		\item Asintoto obliquo
		
		Non esiste asintoto obliquo
	\end{enumerate}
	\litem{Intersezioni}
	\begin{enumerate}
		\item Asse $x$
		\begin{align*}
		&\begin{cases}
		y=0\\
		y=(x^2-1)(x-4)
		\end{cases}&
		&\begin{cases}
		y=0\\
		(x^2-1)(x-4)=0
		\end{cases}
		&
		&\begin{cases}
		y=0\\
		x^2-1=0
		\end{cases}\\
		&\begin{cases}
		y=0\\
		x_1=-1
		\end{cases}&
		&\begin{cases}
		y=0\\
		x_2=1
		\end{cases}&&\begin{cases}
		y=0\\
		x-4=0
		\end{cases}\\
		&\begin{cases}
		y=0\\
		x_3=4
		\end{cases}\\
		\end{align*}
			Quindi abbiamo che la curva interseca l'asse $x$ in tre punti
		\item asse $y$
		
		\begin{align*}
		&\begin{cases}
		x=0\\
		y=(x^2-1)(x-4)
		\end{cases}&
		&\begin{cases}
		x=0\\
		y=(-1)(-4)=4
		\end{cases}\\
	\end{align*}
		Quindi la curva incontra l'asse $y$ in un punto.	
	\end{enumerate}
\litem{Derivata della funzione e segno della derivata}
\begin{align*}
y=&(x^2-1)(x-4)\\
y'=&\OpD{(x^2-1)(x-4)}\\
=&2x(x-4)+x^2-1\\
=&2x^2-8x+x^2-1\\
=&3x^2-8x-1
\intertext{Studio il segno della derivata}
y'=&3x^2-8x-1\geq 0
\intertext{risolvo l'equazione}
3x^2-8x-1=&0\\
x_{1,2}=&\dfrac{8\pm\sqrt{64+12}}{6}\\
=&\dfrac{8\pm\sqrt{76}}{6}\\
=&\dfrac{8\pm\sqrt{2^2\cdot 19}}{6}\\
=&\dfrac{8\pm 2\sqrt{19}}{6}
=\begin{cases}
x_1=&\dfrac{4+\sqrt{19}}{3}\\
x_2=&\dfrac{4-\sqrt{19}}{3}
\end{cases}\\
\end{align*}
Otteniamo il grafico:
\begin{center}
	\includestandalone{quinto/grafici/disgraficoprob12}
\end{center}
Osservando il grafico notiamo che prima di $x=\dfrac{4-\sqrt{19}}{3}$ la derivata è positiva, poi va a zero infine è negativa. Quindi per $x=\dfrac{4-\sqrt{19}}{3}$ la funzione ha un massimo.  Prima di $x=\dfrac{4+\sqrt{19}}{3}$ la derivata è negativa, poi va a zero infine è positiva. Quindi per $x=\dfrac{4-\sqrt{19}}{3}$ la funzione ha un minimo. 
\litem{Flessi e concavità}

Dal segno della derivata prima  vediamo che non ci sono flessi orizzontali.
Per la concavità e gli altri flessi studiamo la derivata seconda
\begin{align*}
y'=&3x^2-8x-1\\
y''=&\OpD{3x^2-8x-1}\\
=&6x-8\\
\intertext{Studiamo il segno della derivata}
6x-8\geq&0\\
x\geq&\dfrac{8}{6}=\dfrac{4}{3}\\
\end{align*}
Otteniamo il grafico:
\begin{center}
	\includestandalone{quinto/grafici/disgraficoprob13}
\end{center}
Da grafico, per valori di $x<\dfrac{4}{3}$ la concavità è rivolta verso il basso. Per valori di $x>\dfrac{4}{3}$   la concavità è rivolta vero l'alto. Per $x=\dfrac{4}{3}$ è un punto di flesso.
	\litem{Grafico probabile}
	\begin{center}
		\includestandalone[width=0.5\textwidth]{quinto/grafici/graficoprobabile4}
	\end{center}
\end{itemize}
\end{exercise}
\begin{exercise}
	Disegnare il grafico probabile di questa funzione $f(x)=(x^2-1)(x^2-4)$
	\tcblower
	Disegnare il grafico probabile di questa funzione $f(x)=(x^2-1)(x^2-4)$

			\begin{itemize}
			\litem {Dominio}
			La funzione è intera quindi è definita su tutto $\R$
			\litem{Positività}
			Studiamo il segno di $f(x)=(x^2-1)(x^2-4)$
			\begin{align*}
			(x^2-1)(x^2-4)\geq&0\\
			\intertext{Risolviamo le due disequazioni}
			x^2-4\geq&0\\
			x^2-4=&0\\
				x_{1,2}=&\dfrac{\pm\sqrt{16}}{2}
			=\begin{cases}
			x_1=+2\\
			x_2=-2
			\end{cases}\\
			x^2-1\geq&0\\
			x^2-1=&0\\
			x_{1,2}=&\dfrac{\pm\sqrt{4}}{2}
			=\begin{cases}
			x_1=+1\\
			x_2=-1
			\end{cases}
			\end{align*}
			Otteniamo il grafico:
			\begin{center}
				\includestandalone{quinto/grafici/disgraficoprob14}
			\end{center}
			Dal diagramma vediamo che la funzione è positiva per valori di $x$ compresi tra meno uno e più uno e per valori di $x$ maggiori di quattro $-1\leq x \leq 1$ o $x\geq 4$
	\end{itemize}
\end{exercise}
%\begin{exercise}[no solution]
%	It holds:
%	\begin{equation*}
%	\frac{d}{dx}\left(\ln|x|\right) = \frac{1}{x}.
%	\end{equation*}
%\end{exercise}

\tcbstoprecording
\newpage
\section{Soluzione esercizi}
\tcbinputrecords