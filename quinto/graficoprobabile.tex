\section{Funzioni razionali fratte}
Risolvi i seguenti esercizi
\tcbstartrecording
%\begin{exercise}

%	\tcblower
%	\begin{itemize}
%	\litem {Dominio}
%	
%	\litem{Positività}
%	
%	Otteniamo il grafico:
%	
%	\litem{Asintoti}
%	\begin{enumerate}
%		\item Asintoti verticali
%		
%		Utilizzando il grafico precedente
%		
%		\item Asintoto orizzontale 
%		
%		\item Asintoto obliquo
%		
%	\end{enumerate}
%	\litem{Intersezioni}
%	\begin{enumerate}
%		\item Asse $x$
%		\item asse $y$
%		
%		\item Asintoti
%		
%	\end{enumerate}
%	\litem{Grafico probabile}
%\end{itemize}
%\end{exercise}
%\begin{exercise}[no solution]
%	It holds:
%	\begin{equation*}
%	\frac{d}{dx}\left(\ln|x|\right) = \frac{1}{x}.
%	\end{equation*}
%\end{exercise}
\begin{exercise}[no solution]
Disegna il grafico probabile di questa funzione $f(x)= \dfrac{2x+1}{3x-2}$
\end{exercise}
\begin{exercise}
	Trovare il grafico probabile della funzione $y=\dfrac{x^2-1}{x+5}$ 
	\tcblower
	Trovare il grafico probabile della funzione $y=\dfrac{x^2-1}{x+5}$ 
	\begin{itemize}
		\litem {Dominio}
		\begin{align*}
		x+5=&0\\
		x=&-5\\
		x\neq&-5	
		\end{align*}
		\litem{Positività}
		\begin{align*}
	\dfrac{x^2-1}{x+5}\geq&0\\
	x+5>&0\\
	x>&-5\\
	x^2-1\geq&0\\
	x^2-1=&0\\
	x_{1,2}=&\dfrac{\pm\sqrt{4}}{2}\\
	=&\begin{cases}
	x_1=+1\\
	x_2=-1
	\end{cases}
		\end{align*}
		Otteniamo il grafico:
		\begin{center}
			\includestandalone{quinto/grafici/disgraficoprob1}
		\end{center}
		\litem{Asintoti}
		\begin{enumerate}
			\item Asintoti verticali
			
	Utilizzando il grafico precedente
	\begin{align*}
	\lim_{x\to -5^-}\dfrac{x^2-1}{x+5}=&-\infty\\
	\lim_{x\to -5^+}\dfrac{x^2-1}{x+5}=&+\infty\\
	x=&-5\\
	\intertext{è asintoto verticale}
	\end{align*}
			\item Asintoto orizzontale 
			
			Dato che il grado del  numeratore è maggiore del grado denominatore l'asintoto non esiste.
			\item Asintoto obliquo
			\begin{align*}
			m=&\lim_{x\to \infty}\dfrac{x^2-1}{x+5}\dfrac{1}{x}\\
			=&\lim_{x\to \infty}\dfrac{x^2-1}{x^2+5x}\\
			=&\lim_{x\to \infty}\dfrac{x^2\left(1-\dfrac{1}{x^2}\right)}{x^2\left(1+\dfrac{5x}{x^2}\right)}=1\\
			q=&\lim_{x\to \infty}[\dfrac{x^2-1}{x+5}-x]\\
			=&\lim_{x\to \infty}\dfrac{x^2-1-x^2-5x}{x+5}\\
			=&\lim_{x\to \infty}\dfrac{-5x-1}{x+5}\\
			=&\lim_{x\to \infty}\dfrac{x\left(-5-\dfrac{1}{x}\right)}{x\left(1+\dfrac{5}{x}\right)}=-5\\
			y=&x-5
			\end{align*}
		\end{enumerate}
	\litem{Intersezioni}
	\begin{enumerate}
		\item Asse $x$
	\begin{align*}
		&\begin{cases}
	y=\dfrac{x^2-1}{x+5}\\
	y=0\\
	\end{cases}&
		&\begin{cases}
	x^2-1=0\\
	y=0\\
	\end{cases}
		&\begin{cases}
	x_1=+1\\
	y=0
	\end{cases}&
		&\begin{cases}
	x_2=-1\\
	y=0
	\end{cases}&
	\end{align*}
		\item asse $y$
			\begin{align*}
		&\begin{cases}
		y=\dfrac{x^2-1}{x+5}\\
		x=0\\
		\end{cases}&
		&\begin{cases}
		y=-\dfrac{1}{5}\\
		x=0\\
		\end{cases}
		\end{align*}
		\item Asintoti
		\begin{align*}
		&\begin{cases}
	y=\dfrac{x^2-1}{x+5}\\
	y=x-5\\
	\end{cases}&&\begin{cases}
	\dfrac{x^2-1}{x+5}=x-5\\
	y=x-5\\
	\end{cases}
	&\begin{cases}
	x^2-1=x^2-25\\
	y=x-5
	\end{cases}
		\end{align*}
		non ha soluzione 
	\end{enumerate}
\litem{Derivata della funzione e segno della derivata}
\begin{align*}
y=&\dfrac{2x+1}{3x-1}\\
y'=&\OpD{\dfrac{2x+1}{3x-1}}\\
=&\dfrac{2(3x-2)-3(2x+1)}{(3x-2)^2}\\
=&\dfrac{x^2+10x+1}{(x+5)^2}
\intertext{Studio il segno della derivata}
y'=&\dfrac{x^2+10x+1}{(x+5)^2}\geq 0\\
\intertext{Dato che il denominatore è un quadrato e quindi sempre positivo, studiamo solo il segno del numeratore}
\intertext{risolvo l'equazione}
x^2+10x+1=&0\\
x_{1,2}=&\dfrac{-10\pm\sqrt{100-4}}{2}\\
=&\dfrac{-10\pm\sqrt{96}}{2}\\
=&\dfrac{-10\pm\sqrt{2^4\cdot6}}{2}\\
=&\dfrac{-10\pm 4\sqrt{6}}{2}\\
=&\begin{cases}
x_1=-5+\sqrt{6}\\
x_2=-5-\sqrt{6}
\end{cases}
\end{align*}
Otteniamo il grafico:
\begin{center}
	\includestandalone{quinto/grafici/disgraficoprob2}
\end{center}
Osservando il grafico notiamo che prima di $x=-5-\sqrt{5}$ la derivata è positiva, poi va a zero infine è negativa quindi per $x=-5-\sqrt{5}$ la funzione ha un massimo.  Per $x=-5+\sqrt{5}$ la derivata è negativa, poi va a zero infine è positiva quindi per $x=-5+\sqrt{5}$ la funzione ha un minimo.
\litem{Flessi e concavità}

Dal segno della derivata prima  vediamo che non ci sono flessi orizzontali.
Per la concavità e gli altri flessi studiamo la derivata seconda
\begin{align*}
y'=&\dfrac{x^2+10x+1}{(x+5)^2}\\
y''=&\OpD{\dfrac{x^2+10x+1}{(x+5)^2}}\\
=&\dfrac{(2x+10)(x+5)^2-2(x^2+10x+1)(x+5)}{(x+5)^4}\\
=&\dfrac{(2x+10)(x+5)-2(x^2+10x+1)}{(x+5)^3}\\
=&\dfrac{2x^2+10x+10x+50-2x^2-20x-2}{(x+5)^3}\\
=&\dfrac{48}{(x+5)^3}\\
\intertext{Studiamo il segno della derivata}
=&\dfrac{48}{(x+5)^3}\geq 0\\
\intertext{il numeratore è positivo}
\intertext{Segno denominatore}
(x+5)^3>&0\\
x+5>&0\\
x>&-5\\
\end{align*}
Otteniamo il grafico:
\begin{center}
	\includestandalone{quinto/grafici/disgraficoprob3}
\end{center}
Da grafico, per valori minori di meno cinque la concavità è rivolta verso il basso. Per valori superiori la funzione ha concavità rivolta vero l'alto. Per $x=-5$ non abbiamo un flesso perché la funzione non esiste.
\litem{Grafico probabile}
	\end{itemize}
	\begin{center}
		\includestandalone[width=0.6\textwidth]{quinto/grafici/graficoprobabile1}
	\end{center}
\end{exercise}
\begin{exercise}
Disegna il grafico probabile di questa funzione $f(x)= \dfrac{4x^2+4x+1}{x^2}$
	\tcblower
Disegna il grafico probabile di questa funzione $f(x)= \dfrac{4x^2+4x+1}{x^2}$
	\begin{itemize}
	\litem {Dominio}
	\begin{align*}
x^2=&0\\
x_{1,2}=0
x&\neq 0\\
	\end{align*}
	\litem{Positività}
	\begin{align*}
	 \dfrac{4x^2+4x+1}{x^2}\geq&0\\
	4x^2+4x+1\geq&0\\
	4x^2+4x+1=&0\\
	x_{1,2}=&\dfrac{-4\pm\sqrt{16-16}}{8}\\
	=&-\dfrac{4}{8}\\
	=&-\dfrac{1}{2}\\
	x^2>&0\\
	\end{align*}
	
	Otteniamo il grafico:
	\begin{center}
		\includestandalone{quinto/grafici/disgraficoprob4}
	\end{center}
	\litem{Asintoti}
	\begin{enumerate}
		\item Asintoti verticali
		
		Utilizzando il grafico precedente
		\begin{align*}
		\lim_{x\to 0^+}\dfrac{4x^2+4x+1}{x^2}=&+\infty\\
		\lim_{x\to 0^-}\dfrac{4x^2+4x+1}{x^2}=&+\infty\\
		x=&0\\
		\intertext{è asintoto verticale}
		\end{align*}
	\item Asintoto orizzontale 
		\begin{align*}
	\lim_{x\to \infty}\dfrac{4x^2+4x+1}{x^2}=&\\
	\lim_{x\to \infty}\dfrac{x^2\left(4-\dfrac{4x}{x^2}+\dfrac{1}{x^2}\right)}{x^2}=&4\\
	y=&4\\
	\intertext{asintoto orizzontale}
		\end{align*}
		\item Asintoto obliquo
		
		Non esistono asintoti obliqui
	\end{enumerate}
	\litem{Intersezioni}
	\begin{enumerate}
		\item Asse $x$
		\begin{align*}
		&\begin{cases}
		y=0\\
		y=\dfrac{4x^2+4x+1}{x^2}
		\end{cases}
		&&\begin{cases}
		y=0\\
		4x^2+4x+1=0
		\end{cases}
		&\begin{cases}
		y=0\\
		x_1=-\dfrac{1}{2}
		\end{cases}
		\end{align*}
		\item asse $y$
	Dato il dominio non esistono intersezioni con l'asse $y$
		\item Asintoti
		\begin{align*}
			&\begin{cases}
		y=-\dfrac{1}{2}\\
		y=\dfrac{4x^2+4x+1}{x^2}
		\end{cases}
		&&\begin{cases}
		y=-\dfrac{1}{2}\\
	\dfrac{4x^2+4x+1}{x^2}=-\dfrac{1}{2}
		\end{cases}\\
		&\begin{cases}
	y=-\dfrac{1}{2}\\
	\dfrac{8x^2+8x+2+x^2}{x^2}=0
	\end{cases}
	&&\begin{cases}
	y=-\dfrac{1}{2}\\
	\dfrac{9x^2+8x+2}{x^2}=0\\
	\end{cases}\\	
	&\begin{cases}
	y=1\\
	x_{1,2}=\dfrac{-8\pm\sqrt{64-72}}{18}
	\end{cases}
\end{align*}
Nessuna intersezione
	\end{enumerate}
% 10/03/2018 :: 22:00:44 :: % 10/03/2018 :: 22:00:44 :: % 10/03/2018 :: 22:00:44 :: 
\litem{Derivata della funzione e segno della derivata}
\begin{align*}
y=&\dfrac{4x^2+4x+1}{x^2}\\
y'=&\OpD{\dfrac{4x^2+4x+1}{x^2}}\\
=&\dfrac{(8x+4)(x^2)-2x(4x^2+4x+1)}{x^4}\\
=&\dfrac{8x^3+4x^2-8x^3-8x^2-2x}{x^4}\\
=&\dfrac{-4x^2-2x}{x^4}\\
=&\dfrac{-4x-2}{x^3}
\intertext{Studio il segno della derivata}
y'=&\dfrac{-4x-2}{x^3}\geq 0\\
\intertext{Dato che il denominatore è una potenza dispari è positivo solo per valori di $x$ maggiori di zero, studiamo  il segno del numeratore}
\intertext{risolvo l'equazione}
-4x-2\geq&0\\
x\leq&-\dfrac{1}{2}
\end{align*}
Otteniamo il grafico:
\begin{center}
	\includestandalone{quinto/grafici/disgraficoprob7}
\end{center}
Osservando il grafico notiamo che prima di $x=-\dfrac{1}{2}$ la derivata è negativa, poi va a zero infine è positiva. Quindi per $x=-\dfrac{1}{2}$ la funzione ha un minimo.  Per $x=0$ la  derivata non esiste.
\litem{Flessi e concavità}

Dal segno della derivata prima  vediamo che non ci sono flessi orizzontali.
Per la concavità e gli altri flessi studiamo la derivata seconda
\begin{align*}
y'=&\dfrac{-4x-2}{x^3}\\
y''=&\OpD{\dfrac{-4x-2}{x^3}}\\
=&\dfrac{-4x^3+12x^3+6x^2}{x^6}\\
=&\dfrac{8x^3+6x^2}{x^6}\\
=&\dfrac{8x+6}{x^4}\\
\intertext{Studiamo il segno della derivata}
&\dfrac{8x+6}{x^4}\geq 0\\
\intertext{il denominatore è positivo}
\intertext{Segno numeratore}
8x+6\geq&0\\
x\geq&-\dfrac{3}{4}\\
\end{align*}
Otteniamo il grafico:
\begin{center}
	\includestandalone{quinto/grafici/disgraficoprob8}
\end{center}
Da grafico, per valori $x\leqslant -\dfrac{3}{4}$ la concavità è rivolta verso l'alto. Per valori superiori la funzione ha concavità rivolta vero il basso. Per $x=0$ non abbiamo un flesso perché la funzione non esiste.

% 10/03/2018 :: 22:00:51 :: % 10/03/2018 :: 22:00:52 :: % 10/03/2018 :: 22:00:53 :: 
	\litem{Grafico probabile}
\end{itemize}
\begin{center}
	\includestandalone[width=\textwidth]{quinto/grafici/graficoprobabile2}
\end{center}
\end{exercise}
\begin{exercise}
	Disegna il grafico probabile di questa funzione $f(x)= \dfrac{x^2-9}{x^2-4}$
	\tcblower
	Disegna il grafico probabile di questa funzione $f(x)= \dfrac{x^2-9}{x^2-4}$
	\begin{itemize}
	\litem {Dominio}
	\begin{align*}
x^2-4=&0\\
x_{1,2}=&\dfrac{\pm\sqrt{16}}{2}
=\begin{cases}
x_1=+2\\
x_2=-2
\end{cases}\\
x&\neq+2\\
x&\neq-2
\end{align*}
	\litem{Positività}
		\begin{align*}
	\dfrac{x^2-9}{x^2-4}\geq&0\\
	x^2-9\geq&0\\
	x^2-9=&0\\
	x_{1,2}=&\dfrac{\pm\sqrt{36}}{2}\\
	=&\dfrac{\pm 6}{2}
	=\begin{cases}
	x_1=+3\\
	x_2=-3
	\end{cases}\\
	x^2-4>&0\\
	x^2-4=&0\\
	&\begin{cases}
	x_1=+2\\
	x_2=-2
	\end{cases}\\
	\end{align*}
	Otteniamo il grafico:
	\begin{center}
		\includestandalone{quinto/grafici/disgraficoprob5}
	\end{center}
	\litem{Asintoti}
	\begin{enumerate}
		\item Asintoti verticali
		
		Utilizzando il grafico precedente
		\begin{align*}
		\lim_{x\to -2^+}\dfrac{x^2-9}{x^2-4}=&+\infty\\
		\lim_{x\to -2^-}\dfrac{x^2-9}{x^2-4}=&-\infty\\
		x=&-2\\
		\intertext{è asintoto verticale}
		\lim_{x\to 2^+} \dfrac{x^2-9}{x^2-4}=&-\infty\\
		\lim_{x\to 2^-} \dfrac{x^2-9}{x^2-4}=&+\infty\\
		x=&+2\\
		\intertext{è asintoto verticale}
		\end{align*}
		\item Asintoto orizzontale 
			\begin{align*}
		\lim_{x\to \infty}\dfrac{x^2-9}{x^2-4}=&\\
		\lim_{x\to \infty}\dfrac{x^2\left(1-\dfrac{9}{x^2}\right)}{x^2\left(1-\dfrac{4}{x^2}\right)}=&1\\
		y=&1\\
		\intertext{asintoto orizzontale}
		\end{align*}
		\item Asintoto obliquo
		
		Non esistono asintoti obliqui
	\end{enumerate}
	\litem{Intersezioni}
	\begin{enumerate}
		\item Asse $x$
			\begin{align*}
		&\begin{cases}
		y=0\\
		y=\dfrac{x^2-9}{x^2-4}
		\end{cases}
		&&\begin{cases}
		y=0\\
		x^2-9=0
		\end{cases}
		&\begin{cases}
		y=0\\
		x_1=+3
		\end{cases}&
		&\begin{cases}
		y=0\\
		x_1=-3
		\end{cases}
		\end{align*}
		\item asse $y$
		\begin{align*}
		&\begin{cases}
		x=0\\
		y=\dfrac{x^2-9}{x^2-4}
		\end{cases}&&\begin{cases}
		x=0\\
		y=\dfrac{9}{4}
		\end{cases}
		\end{align*}
		\item Asintoti
		Non ha intersezione
	\end{enumerate}
\litem{Derivata della funzione e segno della derivata}
\begin{align*}
y=&\dfrac{x^2-9}{x^2-4}\\
y'=&\OpD{\dfrac{x^2-9}{x^2-4}}\\
=&\dfrac{2x(x^2-4)-2x(x^2-9)}{(x^2-4)^2}\\
=&\dfrac{2x^3-8x-2x^3+18x}{(x^2-4)^2}\\
=&\dfrac{10x}{(x^2-4)^2}
\intertext{Studio il segno della derivata}
y'=&\dfrac{10x}{(x^2-4)^2}\geq 0\\
\intertext{Dato che il denominatore è una potenza pari è sempre positivo, studiamo  il segno del numeratore}
\intertext{risolvo l'equazione}
10x\geq&0\\
x\geq&0
\end{align*}
Otteniamo il grafico:
\begin{center}
	\includestandalone{quinto/grafici/disgraficoprob9}
\end{center}
Osservando il grafico notiamo che prima di $x=0$ la derivata è negativa, poi va a zero infine è positiva. Quindi per $x=$ la funzione ha un minimo.  Per $x=\pm 2$ la  derivata non esiste.
\litem{Flessi e concavità}

Dal segno della derivata prima  vediamo che non ci sono flessi orizzontali.
Per la concavità e gli altri flessi studiamo la derivata seconda
\begin{align*}
y'=&\dfrac{10x}{(x^2-4)^2}\\
y''=&\OpD{\dfrac{10x}{(x^2-4)^2}}\\
=&\dfrac{10(x^2-4)^2-40x^2(x^2-4)}{(x^2-4)^4}\\
=&\dfrac{10(x^2-4)(x^2-4-4x^2)}{(x^2-4)^4}\\
=&\dfrac{10(-4-3x^2)}{(x^2-4)^3}\\
\intertext{Studiamo il segno della derivata}
&\dfrac{10(-4-3x^2)}{(x^2-4)^3}\geq 0\\
\intertext{Segno denominatore. Dato che è  una potenza dispari studiamo il segno della base}
(x^2-4)^3>&0\\
x^2-4>&0\\
x^2-4=&0\\
x_{1,2}=&\dfrac{\pm\sqrt{16}}{2}
=\begin{cases}
x_1=+2\\
x_2=-2
\end{cases}\\
\intertext{Segno numeratore}
-4-3x^2\geq&0\\
-4-3x^2=&0\\
x_{1,2}=&\dfrac{\pm\sqrt{-48}}{-6}
\intertext{Non ha soluzione}
\end{align*}
Otteniamo il grafico:
\begin{center}
	\includestandalone{quinto/grafici/disgraficoprob10}
\end{center}
Da grafico, per valori di $x$ minori di meno due la concavità è rivolta verso il basso. Per valori di $x$ compresi tra meno due e due  la concavità è rivolta vero l'alto. Per $x$ maggiori di due la concavità rivolta vero il basso. 
Per valori uguali a più e meno due la derivata non esiste.

	\litem{Grafico probabile}
\end{itemize}
\begin{center}
	\includestandalone[width=\textwidth]{quinto/grafici/graficoprobabile3}
\end{center}
\end{exercise}

\newpage
\tcbstoprecording
\section{Soluzione esercizi}
\tcbinputrecords