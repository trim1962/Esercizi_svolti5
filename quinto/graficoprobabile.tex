\chapter{Grafico probabile}
\section{Funzioni razionali fratte}
Risolvi i seguenti esercizi
\tcbstartrecording
%\begin{exercise}

%	\tcblower
%	\begin{itemize}
%	\litem {Dominio}
%	
%	\litem{Positività}
%	
%	Otteniamo il grafico:
%	
%	\litem{Asintoti}
%	\begin{enumerate}
%		\item Asintoti verticali
%		
%		Utilizzando il grafico precedente
%		
%		\item Asintoto orizzontale 
%		
%		\item Asintoto obliquo
%		
%	\end{enumerate}
%	\litem{Intersezioni}
%	\begin{enumerate}
%		\item Asse $x$
%		\item asse $y$
%		
%		\item Asintoti
%		
%	\end{enumerate}
%	\litem{Grafico probabile}
%\end{itemize}
%\end{exercise}
%\begin{exercise}[no solution]
%	It holds:
%	\begin{equation*}
%	\frac{d}{dx}\left(\ln|x|\right) = \frac{1}{x}.
%	\end{equation*}
%\end{exercise}
\begin{exercise}[no solution]
Disegna il grafico probabile di questa funzione $f(x)= \dfrac{2x+1}{3x-2}$
\end{exercise}
\begin{exercise}
	Trovare il grafico probabile della funzione $y=\dfrac{x^2-1}{x+5}$ 
	\tcblower
	Trovare il grafico probabile della funzione $y=\dfrac{x^2-1}{x+5}$ 
	\begin{itemize}
		\litem {Dominio}
		\begin{align*}
		x+5=&0\\
		x=&-5\\
		x\neq&-5	
		\end{align*}
		\litem{Positività}
		\begin{align*}
	\dfrac{x^2-1}{x+5}\geq&0\\
	x+5>&0\\
	x>&-5\\
	x^2-1\geq&0\\
	x^2-1=&0\\
	x_{1,2}=&\dfrac{\pm\sqrt{4}}{2}\\
	=&\begin{cases}
	x_1=+1\\
	x_2=-1
	\end{cases}
		\end{align*}
		Otteniamo il grafico:
		\begin{center}
			\includestandalone{quinto/grafici/dissecondoprimo}
		\end{center}
		\litem{Asintoti}
		\begin{enumerate}
			\item Asintoti verticali
			
	Utilizzando il grafico precedente
	\begin{align*}
	\lim_{x\to -5^-}\dfrac{x^2-1}{x+5}=&-\infty\\
	\lim_{x\to -5^+}\dfrac{x^2-1}{x+5}=&+\infty\\
	x=&-5\\
	\intertext{è asintoto verticale}
	\end{align*}
			\item Asintoto orizzontale 
			
			Dato che il grado del  numeratore è maggiore del grado denominatore l'asintoto non esiste.
			\item Asintoto obliquo
			\begin{align*}
			m=&\lim_{x\to \infty}\dfrac{x^2-1}{x+5}\dfrac{1}{x}\\
			=&\lim_{x\to \infty}\dfrac{x^2-1}{x^2+5x}\\
			=&\lim_{x\to \infty}\dfrac{x^2\left(1-\dfrac{1}{x^2}\right)}{x^2\left(1+\dfrac{5x}{x^2}\right)}=1\\
			q=&\lim_{x\to \infty}[\dfrac{x^2-1}{x+5}-x]\\
			=&\lim_{x\to \infty}\dfrac{x^2-1-x^2-5x}{x+5}\\
			=&\lim_{x\to \infty}\dfrac{-5x-1}{x+5}\\
			=&\lim_{x\to \infty}\dfrac{x\left(-5-\dfrac{1}{x}\right)}{x\left(1+\dfrac{5}{x}\right)}=-5\\
			y=&x-5
			\end{align*}
		\end{enumerate}
	\litem{Intersezioni}
	\begin{enumerate}
		\item Asse $x$
	\begin{align*}
		&\begin{cases}
	y=\dfrac{x^2-1}{x+5}\\
	y=0\\
	\end{cases}&
		&\begin{cases}
	x^2-1=0\\
	y=0\\
	\end{cases}\\
		&\begin{cases}
	x_1=+1\\
	y=0
	\end{cases}&
		&\begin{cases}
	x_2=-1\\
	y=0
	\end{cases}&
	\end{align*}
		\item asse $y$
			\begin{align*}
		&\begin{cases}
		y=\dfrac{x^2-1}{x+5}\\
		x=0\\
		\end{cases}&
		&\begin{cases}
		y=-\dfrac{1}{5}\\
		x=0\\
		\end{cases}
		\end{align*}
		\item Asintoti
		\begin{align*}
		&\begin{cases}
	y=\dfrac{x^2-1}{x+5}\\
	y=x-5\\
	\end{cases}&&\begin{cases}
	\dfrac{x^2-1}{x+5}=x-5\\
	y=x-5\\
	\end{cases}\\
	&\begin{cases}
	x^2-1=x^2-25\\
	y=x-5
	\end{cases}
		\end{align*}
		non ha soluzione 
	\end{enumerate}
\litem{Grafico probabile}
	\end{itemize}
	\begin{center}
		\includestandalone[width=0.6\textwidth]{quinto/grafici/graficoprobabile1}
	\end{center}
\end{exercise}
\begin{exercise}
Disegna il grafico probabile di questa funzione $f(x)= \dfrac{x^2-6x+8}{x^2-x}$
	\tcblower
Disegna il grafico probabile di questa funzione $f(x)= \dfrac{x^2-6x+8}{x^2-x}$
	\begin{itemize}
	\litem {Dominio}
	\begin{align*}
x^2-x=&0\\
X_{1,2}=\dfrac{1\pm\sqrt{1}}{2}\\
&\begin{cases}
x_1=0\\
x_2=1
\end{cases}\\
x&\neq 0\\
x&\neq 1
	\end{align*}
	\litem{Positività}
	\begin{align*}
	\dfrac{x^2-6x+8}{x^2-x}\geq&0\\
	x^2-6x+8\geq&0\\
	x^2-6x+8=&0\\
	x_{1,2}=&\dfrac{6\pm\sqrt{36-32}}{2}\\
	=&\dfrac{6\pm 2}{2}\\
	=&\begin{cases}
	x_1=4\\
	x_2=2
	\end{cases}\\
	x^2-x>&0\\
	x^2-x=&0\\
	&\begin{cases}
	x_1=0\\
	x_2=1
	\end{cases}\\
	\end{align*}
	
	Otteniamo il grafico:
	\begin{center}
		\includestandalone{quinto/grafici/sgsg}
	\end{center}
	\litem{Asintoti}
	\begin{enumerate}
		\item Asintoti verticali
		\begin{align*}
		\lim_{x\to 0^+}\dfrac{x^2-6x+8}{x^2-x}=&-\infty\\
		\lim_{x\to 0^-}\dfrac{x^2-6x+8}{x^2-x}=&+\infty\\
		x=&0\\
		\intertext{è asintoto verticale}
		\lim_{x\to 1^+}\dfrac{x^2-6x+8}{x^2-x}=&+\infty\\
		\lim_{x\to 1^-}\dfrac{x^2-6x+8}{x^2-x}=&-\infty\\
		x=&1\\
		\intertext{è asintoto verticale}
		\end{align*}
		Utilizzando il grafico precedente
		
		\item Asintoto orizzontale 
		\begin{align*}
	\lim_{x\to \infty}\dfrac{x^2-6x+8}{x^2-x}=&\\
	\lim_{x\to \infty}\dfrac{x^2\left(1-\dfrac{6x}{x^2}+\dfrac{8}{x^2}\right)}{x^2\left(1-\dfrac{x}{x^2}\right)}=&1\\
	y=&1\\
	\intertext{asintoto orizzontale}
		\end{align*}
		\item Asintoto obliquo
		
		Non esistono asintoti obliqui
	\end{enumerate}
	\litem{Intersezioni}
	\begin{enumerate}
		\item Asse $x$
		\item asse $y$
		
		\item Asintoti
		
	\end{enumerate}
	\litem{Grafico probabile}
\end{itemize}
\end{exercise}





\tcbstoprecording
\newpage
\section{Soluzione esercizi}
\tcbinputrecords
