\chapter{Limiti}
\section{Limite infinito per x che tende a valore finito}
Risolvi i seguenti limiti
\tcbstartrecording
%\begin{exercise}

%	\tcblower

%\end{exercise}
	%\begin{exercise}[no solution]
%	It holds:
%	\begin{equation*}
%	\frac{d}{dx}\left(\ln|x|\right) = \frac{1}{x}.
%	\end{equation*}
%\end{exercise}
\begin{exercise}
\begin{equation*}
\lim_{x\to 2^+}\dfrac{1}{x-2}
\end{equation*}
	\tcblower
Sostituendo due al denominatore otteniamo uno diviso zero. Possiamo risolvere l'esercizio in due modi. Primo metodo intuitivo, se sostituisco nel denominatore valori leggermente superiori al due la differenza è positiva quindi \begin{equation*}
\lim_{x\to 2^+}\dfrac{1}{x-2}=+\infty
\end{equation*} Secondo metodo analitico. Studio con una disequazione il segno della frazione
\[x-2>0 \] 
ottengo il grafico
\begin{center}
	\includestandalone{quinto/grafici/disprimogrado}
\end{center}\index{Limite!infinito!tende finito}
Quindi la frazione è positiva per valori a destra del due. Segue che 
\begin{equation*}
\lim_{x\to 2^+}\dfrac{1}{x-2}=+\infty
\end{equation*}
\end{exercise}
\begin{exercise}
	\begin{equation*}
	\lim_{x\to -5^-}\dfrac{x-3}{x+5}
	\end{equation*}
	\tcblower
	Sostituendo meno cinque al denominatore otteniamo meno otto diviso zero. Possiamo risolvere l'esercizio in due modi. Primo metodo intuitivo, sostituendo meno cinque all'incognita  nel numeratore otteniamo un valore negativo. Procedendo in modo analogo al denominatore otteniamo un valore negativo (attenzione che consideriamo valori a sinistra di meno cinque). Quindi in numeratore e il denominatore sono entrambi negativi per cui la frazione a sinistra è positiva. Segue \begin{equation*}
	\lim_{x\to -5^-}\dfrac{x-3}{x+5}=+\infty
	\end{equation*} Secondo metodo analitico. Studio con una disequazione il segno della frazione
	\begin{align*}
	x-3>&0\\
	x>&3\\
	x+5>&0\\
	x>&-5
	\end{align*} 
	ottengo il grafico
	\begin{center}
		\includestandalone{quinto/grafici/disprimogradodue}
	\end{center}\index{Limite!infinito!tende finito}
	Quindi la frazione è positiva per valori a sinistra di meno cinque. Segue che 
	\begin{equation*}
		\lim_{x\to -5^-}\dfrac{x-3}{x+5}=+\infty
	\end{equation*}
\end{exercise}
	\begin{exercise}[no solution]
	\begin{equation*}
	\lim_{x\to -3}\dfrac{2}{(x+3)^2}
	\end{equation*}
\end{exercise}
	\begin{exercise}[no solution]
	\begin{equation*}
	\lim_{x\to -2}\dfrac{2x}{(x+2)^2}
	\end{equation*}
\end{exercise}
	\begin{exercise}[no solution]
	\begin{equation*}
	\lim_{x\to 0^-}\left(\dfrac{1}{x^3}+x\right)
	\end{equation*}
\end{exercise}

\section{Limite infinito per x che tende ad infinito}
Risolvere i seguenti limiti
\begin{exercise}
\begin{equation*}
\lim_{x\to +\infty}3x^3-x^2-x-1
\end{equation*}
	\tcblower
Per valori di $x$ che tende all'infinito le varie parti del polinomio tendono a più o meno infinito. Per ovviare a questa contraddizione si procede come segue.
\begin{align*}
\lim_{x\to +\infty}3x^3-x^2-x-1=&\\
&=\lim_{x\to +\infty} x^3\left(3-\dfrac{x^2}{x^3}-\dfrac{x}{x^3}-\dfrac{1}{x^3}\right)\\
&=\lim_{x\to +\infty} x^3\left(3-\dfrac{1}{x}-\dfrac{1}{x^2}-\dfrac{1}{x^3}\right)
\intertext{i termini all'interno della parentezi tendono a tre ora mentre il termine all'esterno tende a più infinito quindi}
=&+\infty
\end{align*}\index{Limite!infinito!tende infinito}
\end{exercise}
	\begin{exercise}[no solution]
	\begin{equation*}
	\lim_{x\to +\infty}3x^5+x^2-x+8
	\end{equation*}
\end{exercise}
\section{Forma indeterminata zero su zero}
Risolvere i seguenti limiti
\begin{exercise}
\begin{equation*}
\lim_{x\to 1}\dfrac{x^2-1}{x-1}
\end{equation*}
	\tcblower
\begin{align*}
\lim_{x\to 1}\dfrac{x^2-1}{x-1}=&\\
&=\lim_{x\to 1}\dfrac{(x-1)(x+1)}{x-1}
\intertext{semplificando}
&=\lim_{x\to 1}(x+1)\\
&=2
\end{align*}
\end{exercise}
\begin{exercise}
\begin{equation*}
\lim_{x\to -2}\dfrac{2x^2+x-6}{4x^2+9x+2}
\end{equation*}
	\tcblower
\begin{equation*}
\lim_{x\to -2}\dfrac{2x^2+x-6}{4x^2+9x+2}=\dfrac{0}{0}
\end{equation*}
Per risolvere questa forma indeterminata scompongo
il numeratore
\begin{align*}
2x^2+x-6=&0\\
x_{1,2}=&\dfrac{-1\pm\sqrt{1+48}}{4}\\
=&\dfrac{1\pm 7}{4}\\
=&\begin{cases}
x_1=-2\\
x_2=-\dfrac{3}{2}
\end{cases}\\
2x^2+x-6=&2(x+\dfrac{3}{2})(x+2)
\end{align*}
scompongo
il denominatore
\begin{align*}
4x^2+9x+2=&0\\
x_{1,2}=&\dfrac{-9\pm\sqrt{81-32}}{8}\\
=&\dfrac{-9\pm 7}{8}\\
=&\begin{cases}
x_1=-2\\
x_2=-\dfrac{1}{4}
\end{cases}\\
4x^2+9x+2=&4(x+\dfrac{1}{4})(x+2)
\end{align*}
quindi
\begin{align*}
\lim_{x\to -2}\dfrac{2x^2+x-6}{4x^2+9x+2}=&\\
=&\lim_{x\to -2}\dfrac{2(x+\dfrac{3}{2})(x+2)}{4(x+\dfrac{1}{4})(x+2)}
\intertext{semplificando}
=&\lim_{x\to -2}\dfrac{2(x+\dfrac{3}{2})}{4(x+\dfrac{1}{4})}\\
=&\lim_{x\to -2}\dfrac{2x+3}{4x+4}\\
=\dfrac{1}{4}
\end{align*}
\end{exercise}
\tcbstoprecording
\newpage
\section{Soluzioni limiti}
\tcbinputrecords
\newpage