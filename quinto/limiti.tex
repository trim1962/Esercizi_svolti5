\chapter{Limiti}
\section{Limite infinito per x che tende a valore finito}
\tcbstartrecording
%\begin{exercise}
%	\tcblower
%\end{exercise}
	%\begin{exercise}[no solution]
%	It holds:
%	\begin{equation*}
%	\frac{d}{dx}\left(\ln|x|\right) = \frac{1}{x}.
%	\end{equation*}
%\end{exercise}
\begin{exercise}
Calcoliamo il seguente limite
$\lim_{x\to 2^+}\dfrac{1}{x-2}$
	\tcblower
Calcoliamo il seguente limite
$\lim_{x\to 2^+}\dfrac{1}{x-2}$
	
Sostituendo due al denominatore otteniamo uno diviso zero. Possiamo risolvere l'esercizio in due modi. Primo metodo intuitivo, se sostituiamo all'incognita nel denominatore, valori leggermente superiori al due la differenza è positiva quindi \begin{equation*}
\lim_{x\to 2^+}\dfrac{1}{x-2}=+\infty
\end{equation*} Secondo metodo analitico. Studiamo, con una disequazione, il segno della frazione
\[x-2>0 \] 
otteniamo il grafico
\begin{center}
	\includestandalone{quinto/grafici/disprimogrado}
\end{center}\index{Limite!infinito!tende finito}
Quindi la frazione è positiva per valori a destra del due. Segue che 
\begin{equation*}
\lim_{x\to 2^+}\dfrac{1}{x-2}=+\infty
\end{equation*}
\end{exercise}
\begin{exercise}
Calcoliamo il limite 
	$\lim_{x\to -5^-}\dfrac{x-3}{x+5}$
	\tcblower
	Calcoliamo il limite 
	$\lim_{x\to -5^-}\dfrac{x-3}{x+5}$
	Sostituendo meno cinque al denominatore otteniamo meno otto diviso zero. Possiamo risolvere l'esercizio in due modi. Primo metodo intuitivo, sostituendo meno cinque all'incognita nel numeratore otteniamo un valore negativo. Procedendo in modo analogo al denominatore otteniamo un valore negativo (attenzione che consideriamo valori a sinistra di meno cinque). Quindi in numeratore e il denominatore sono entrambi negativi per cui la frazione a sinistra è positiva. Segue \begin{equation*}
	\lim_{x\to -5^-}\dfrac{x-3}{x+5}=+\infty
	\end{equation*} Secondo metodo analitico. Studiamo, con una disequazione, il segno della frazione
	\begin{align*}
	x-3>&0\\
	x>&3\\
	x+5>&0\\
	x>&-5
	\end{align*} 
	otteniamo il grafico
	\begin{center}
		\includestandalone{quinto/grafici/disprimogradodue}
	\end{center}\index{Limite!infinito!tende finito}
	Quindi la frazione è positiva per valori a sinistra di meno cinque. Segue che 
	\begin{equation*}
		\lim_{x\to -5^-}\dfrac{x-3}{x+5}=+\infty
	\end{equation*}
\end{exercise}
	\begin{exercise}[no solution]
Calcoliamo il limite
	$\lim_{x\to -3}\dfrac{2}{(x+3)^2}$

\end{exercise}
	\begin{exercise}[no solution]
Calcoliamo il limite
	$\lim_{x\to -2}\dfrac{2x}{(x+2)^2}$
\end{exercise}
	\begin{exercise}[no solution]
	Calcoliamo il limite
	$\lim_{x\to 0^-}\left(\dfrac{1}{x^3}+x\right)$
\end{exercise}
\tcbstoprecording
\newpage
\section{Soluzione esercizi}
\tcbinputrecords
\newpage
\section{Limite infinito per x che tende ad infinito}
\tcbstartrecording
\begin{exercise}
Calcoliamo il limite 
$\lim_{x\to +\infty}3x^3-x^2-x-1$
	\tcblower
	Calcoliamo il limite 
	$\lim_{x\to +\infty}3x^3-x^2-x-1$
Per valori di $x$ che tendono all'infinito le varie parti del polinomio tendono a più o meno infinito. Per ovviare a questa contraddizione procediamo come segue:
\begin{align*}
\lim_{x\to +\infty}3x^3-x^2-x-1=&\\
&=\lim_{x\to +\infty} x^3\left(3-\dfrac{x^2}{x^3}-\dfrac{x}{x^3}-\dfrac{1}{x^3}\right)\\
&=\lim_{x\to +\infty} x^3\left(3-\dfrac{1}{x}-\dfrac{1}{x^2}-\dfrac{1}{x^3}\right)
\intertext{i termini all'interno della parentesi tendono a tre mentre il termine all'esterno tende a più infinito quindi}
=&+\infty
\end{align*}\index{Limite!infinito!tende infinito}
\end{exercise}
	\begin{exercise}[no solution]
Calcoliamo il limite
$\lim_{x\to +\infty}3x^5+x^2-x+8$
\end{exercise}
\tcbstoprecording
\newpage
\section{Soluzione esercizi}
\tcbinputrecords
\newpage
\section{Forma indeterminata zero su zero}
\tcbstartrecording
\begin{exercise}
Calcoliamo il limite
$\lim_{x\to 1}\dfrac{x^2-1}{x-1}$
	\tcblower
Calcoliamo il limite
$\lim_{x\to 1}\dfrac{x^2-1}{x-1}$	
\begin{align*}
\lim_{x\to 1}\dfrac{x^2-1}{x-1}=\dfrac{0}{0}&
\intertext{scomponiamo il numeratore}
&=\lim_{x\to 1}\dfrac{(x-1)(x+1)}{x-1}
\intertext{semplificando}
&=\lim_{x\to 1}(x+1)\\
&=2
\end{align*}
\end{exercise}
\begin{exercise}
Calcoliamo il limite
$\lim_{x\to -2}\dfrac{2x^2+x-6}{4x^2+9x+2}$
	\tcblower
	Calcoliamo il limite
	$\lim_{x\to -2}\dfrac{2x^2+x-6}{4x^2+9x+2}$
	\begin{equation*}
\lim_{x\to -2}\dfrac{2x^2+x-6}{4x^2+9x+2}=\dfrac{0}{0}
\end{equation*}
Per risolvere questa forma indeterminata scomponiamo
il numeratore
\begin{align*}
2x^2+x-6=&0\\
x_{1,2}=&\dfrac{-1\pm\sqrt{1+48}}{4}\\
=&\dfrac{1\pm 7}{4}\\
=&\begin{cases}
x_1=-2\\
x_2=\dfrac{3}{2}
\end{cases}\\
2x^2+x-6=&2(x-\dfrac{3}{2})(x+2)
\intertext{scomponiamo
il denominatore}
4x^2+9x+2=&0\\
x_{1,2}=&\dfrac{-9\pm\sqrt{81-32}}{8}\\
=&\dfrac{-9\pm 7}{8}\\
=&\begin{cases}
x_1=-2\\
x_2=-\dfrac{1}{4}
\end{cases}\\
4x^2+9x+2=&4(x+\dfrac{1}{4})(x+2)
\intertext{quindi}
\lim_{x\to -2}\dfrac{2x^2+x-6}{4x^2+9x+2}=&\\
=&\lim_{x\to -2}\dfrac{2(x-\dfrac{3}{2})(x+2)}{4(x+\dfrac{1}{4})(x+2)}
\intertext{semplificando}
=&\lim_{x\to -2}\dfrac{2(x-\dfrac{3}{2})}{4(x+\dfrac{1}{4})}\\
=&\lim_{x\to -2}\dfrac{2x+3}{4x+1}\\
=&\dfrac{1}{7}
\end{align*}
\end{exercise}
\begin{exercise}
Calcoliamo il limite
$\lim_{x\to 1}\dfrac{2x^2+x-3}{x^2-x}$
	\tcblower
	Calcoliamo il limite
	$\lim_{x\to 1}\dfrac{2x^2+x-3}{x^2-x}$
\begin{equation*}
\lim_{x\to 1}\dfrac{2x^2+x-3}{x^2-x}=\dfrac{0}{0}
\end{equation*}
Per risolvere questa forma indeterminata scomponiamo
il numeratore
\begin{align*}
2x^2+x-3=&0\\ 
x_{1,2}=&\dfrac{-1\pm\sqrt{1+24}}{4}\\
=&\dfrac{1\pm 5}{4}\\
=&\begin{cases}
x_1=1\\
x_2=-\dfrac{3}{2}
\end{cases}\\
2x^2+x-3=&2(x+\dfrac{3}{2})(x-1)
\intertext{scomponiamo il denominatore}
x^2-x=&0\\
x_{1,2}=&\dfrac{1\pm\sqrt{1}}{2}\\
=&\dfrac{1\pm 1}{2}\\
=&\begin{cases}
x_1=0\\
x_2=1
\end{cases}\\
x^2-x=&x(x-1)
\intertext{quindi}
\lim_{x\to 1}\dfrac{2x^2+x-3}{x^2-x}=&\\
=&\lim_{x\to 1}\dfrac{2(x+\dfrac{3}{2})(x-1)}{x(x-1)}
\intertext{semplificando}
=&\lim_{x\to 1}\dfrac{2(x+\dfrac{3}{2})}{x}\\
=&\lim_{x\to 1}\dfrac{2x+3}{x}\\
=&5
\end{align*}
\end{exercise}
	\begin{exercise}[no solution]
	Calcoliamo il limite
	$\lim_{x\to 1}\dfrac{2x^2+5x-6}{x^2+x-2}$
\end{exercise}
\tcbstoprecording
\newpage
\section{Soluzione esercizi}
\tcbinputrecords
\newpage
\section{Forma indeterminata infinito su infinito}
\tcbstartrecording
\begin{exercise}
Calcoliamo il limite
$\lim_{x\to +\infty}\dfrac{2x-3}{1-5x}$
	\tcblower
	Calcoliamo il limite
	$\lim_{x\to +\infty}\dfrac{2x-3}{1-5x}$ Limite del tipo infinito su infinito, procediamo riscrivendo la frazione 
	\begin{align*}
\lim_{x\to +\infty}\dfrac{2x-3}{1-5x}=&\\
&=\lim_{x\to +\infty}\dfrac{x\left(2-\dfrac{3}{x}\right)}{x\left(\dfrac{1}{x}-5\right)}
\intertext{semplifico}
&=\lim_{x\to +\infty}\dfrac{2-\dfrac{3}{x}}{\dfrac{1}{x}-5}\\
&=-\dfrac{2}{5}
	\end{align*}
\end{exercise}
\begin{exercise}
Calcoliamo il limite
	$\lim_{x\to +\infty}\dfrac{x^2-3x+4}{x^2+x-6}$
	\tcblower
	Calcoliamo il limite
	$\lim_{x\to +\infty}\dfrac{x^2-3x+4}{x^2+x-6}$ Limite del tipo infinito su infinito, procediamo riscrivendo la frazione 
	\begin{align*}
	\lim_{x\to +\infty}\dfrac{x^2-3x+4}{x^2+x-6}=&\\
	&=\lim_{x\to +\infty}\dfrac{x^2\left(1-\dfrac{3x}{x^2}+\dfrac{4}{x^2}\right)}{x^2\left(1+\dfrac{x}{x^2}-\dfrac{6}{x^2}\right)}
	\intertext{semplifico}
	&=\lim_{x\to +\infty}\dfrac{1-\dfrac{3}{x}+\dfrac{4}{x^2}}{1+\dfrac{1}{x}-\dfrac{6}{x^2}}\\
	&=1
	\end{align*}
\end{exercise}
\begin{exercise}
Calcoliamo il limite 
$\lim_{x\to +\infty}\dfrac{x^2-x+1}{x^4-3x+4}$
\tcblower
Calcoliamo il limite 
$\lim_{x\to +\infty}\dfrac{x^2-x+1}{x^4-3x+4}$ Limite del tipo infinito su infinito, procediamo riscrivendo la frazione 
\begin{align*}
\lim_{x\to +\infty}\dfrac{x^2-x+1}{x^4-3x+4}=&\\
&=\lim_{x\to +\infty}\dfrac{x^2\left(1-\dfrac{x}{x^2}+\dfrac{1}{x^2}\right)}{x^4\left(1-\dfrac{3x}{x^4}+\dfrac{4}{x^4}\right)}
\intertext{semplifico}
&=\lim_{x\to +\infty}\dfrac{1-\dfrac{1}{x}+\dfrac{1}{x^2}}{x^2\left(1-\dfrac{3}{x^3}+\dfrac{4}{x^4}\right)}\\
&=0
\end{align*}
\end{exercise}
\begin{exercise}
Calcoliamo il limite 
	$\lim_{x\to +\infty}\dfrac{x^5-3x+1}{2x+1}$
	\tcblower
	Calcoliamo il limite 
	$\lim_{x\to +\infty}\dfrac{x^5-3x+1}{2x+1}$ Limite del tipo infinito su infinito, procediamo riscrivendo la frazione 
	\begin{align*}
	\lim_{x\to +\infty}\dfrac{x^5-3x+1}{2x+1}=&\\
	&=\lim_{x\to +\infty}\dfrac{x^5\left(1-\dfrac{3x}{x^5}+\dfrac{1}{x^5}\right)}{x\left(2+\dfrac{1}{x}\right)}
	\intertext{semplifico}
	&=\lim_{x\to +\infty}\dfrac{x^4\left(1-\dfrac{3}{x^4}+\dfrac{1}{x^5}\right)}{2+\dfrac{1}{x}}\\
	&=\infty
	\end{align*}
\end{exercise}
\tcbstoprecording
\newpage
\section{Soluzione esercizi}
\tcbinputrecords
\newpage

