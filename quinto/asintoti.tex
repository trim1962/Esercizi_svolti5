\chapter{Asintoti}
\section{Asintoti verticali}
Risolvi i seguenti esercizi
%\tcbstartrecording
%\begin{exercise}
%	\tcblower
%\end{exercise}
	%\begin{exercise}[no solution]
%	It holds:
%	\begin{equation*}
%	\frac{d}{dx}\left(\ln|x|\right) = \frac{1}{x}.
%	\end{equation*}
%\end{exercise}
\begin{exercise}[no solution]
	$f(x)= \dfrac{2x^2+1}{x-1}$
\end{exercise}
\begin{exercise}
 $y=\dfrac{x^2-5x+6}{x^2-1}$
	\tcblower
	$y=\dfrac{x^2-5x+6}{x^2-1}$
	
Studiamo il dominio della funzione. 
\begin{align*}
x^2-1=&0\\ 
x_{1,2}=&\dfrac{0\pm\sqrt{0+4}}{2}\\
=&\dfrac{\pm 2}{2}\\
=&\begin{cases}
x_1=+1\\
x_2=-1
\end{cases}\\
\end{align*}
Quindi il dominio della funzione è  $x\neq\pm1$

Calcoliamo il limite
\begin{align*}
\lim_{x\to 1^+}\frac{x^2-5x+6}{x^2-1}=&\\
\intertext{Dato che}
(1^+)^2-1>&0\\
(1^+)^2-5\cdot 1^++6>&0\\
\intertext{otteniamo}
\lim_{x\to 1^+}\frac{x^2-5x+6}{x^2-1}=&+\infty\\
\intertext{Calcoliamo il limite}
\lim_{x\to 1^-}\frac{x^2-5x+6}{x^2-1}=&\\
\intertext{Dato che}
(1^-)^2-1<&0\\
(1^-)^2-5\cdot 1^++6>&0\\
\intertext{otteniamo}
\lim_{x\to 1^+}\frac{x^2-5x+6}{x^2-1}=&+\infty\\
\end{align*} 
quindi $x=1$ è un asintoto verticale.
Calcoliamo il limite
\begin{align*}
\lim_{x\to -1^+}\frac{x^2-5x+6}{x^2-1}=&\\
\intertext{Dato che}
(-1^+)^2-1<&0\\
(-1^+)^2-5\cdot (-1^+)+6>&0\\
\intertext{otteniamo}
\lim_{x\to -1^+}\frac{x^2-5x+6}{x^2-1}=&-\infty\\
\intertext{Calcoliamo il limite}
\lim_{x\to -1^-}\frac{x^2-5x+6}{x^2-1}=&\\
\intertext{Dato che}
(-1^-)^2-1>&0\\
(-1^-)^2-5\cdot (-1^-)+6>&0\\
\intertext{otteniamo}
\lim_{x\to -1^-}\frac{x^2-5x+6}{x^2-1}=&+\infty\\
\end{align*} 
quindi $x=-1$ è un asintoto verticale.
\end{exercise}
\begin{exercise}[no solution]
$f(x)= \dfrac{x^2+1}{x^2-4}$
\end{exercise}
\begin{exercise}
	Trovare gli asintoti richiesti
$y=\dfrac{3x^2+5x+2}{x^2-1}$
	\tcblower
	$y=\dfrac{3x^2+5x+2}{x^2-1}$
	
Calcolo il dominio
\begin{align*}
x^2-1=&0\\ 
x_{1,2}=&\dfrac{0\pm\sqrt{0+4}}{2}\\
=&\dfrac{\pm 2}{2}\\
=&\begin{cases}
x_1=+1\\
x_2=-1
\end{cases}\\
\end{align*}
Quindi il dominio della funzione è  $x\neq\pm1$

Calcoliamo il limite 
\begin{equation*}\lim_{x\to -1}\frac{3x^2+5x+2}{x^2-1}=\dfrac{0}{0}
\end{equation*}
Per risolvere questa forma indeterminata scompongo
il numeratore
\begin{align*}
3x^2+5x+2=&0\\
x_{1,2}=&\dfrac{-5\pm\sqrt{25-24}}{6}\\
=&\dfrac{-5\pm 1}{6}\\
=&\begin{cases}
x_1=-1\\
x_2=-\dfrac{2}{3}
\end{cases}\\
3x^2+5x+2=&3(x+\dfrac{2}{3})(x+1)
\intertext{quindi}
\lim_{x\to -1}\dfrac{3x^2+5x+2}{x^2-1}=&\\
=&\lim_{x\to -1}\dfrac{3(x+\dfrac{2}{3})(x+1)}{(x-1)(x+1)}
\intertext{semplifico}
=&\lim_{x\to -1}\dfrac{3x+2}{x-1}\\
=&\dfrac{1}{2}
\end{align*}
Quindi $x=-1$ non è asintoto verticale. Utilizzando i calcoli precedenti otteniamo il limite:
\begin{equation*}
\lim_{x\to 1}\dfrac{3x^2+5x+2}{x^2-1}=\lim_{x\to 1}\dfrac{3x+2}{x-1}
\end{equation*}
Studiamo con una disequazione il segno della funzione
\begin{equation*}
y=\dfrac{3x+2}{x-1}\geq0
\end{equation*}
\begin{align*}
3x+2\geq&0\\
x\geq&-\dfrac{2}{3}\\
x-1>&0\\
x>&1
\end{align*}
otteniamo il grafico
\begin{center}
	\includestandalone{quinto/grafici/disprimogradodueb}
\end{center}\index{Limite!infinito!tende finito}
Quindi la frazione è positiva per valori a destra di uno e negativa a sinistra. Segue che
\begin{align*}
\lim_{x\to 1^+}\dfrac{3x+2}{x-1}=&+\infty\\
\lim_{x\to 1^-}\dfrac{3x+2}{x-1}=&-\infty\\
\end{align*}
Quindi $x=1$ è asintoto verticale\index{Asintoto!verticale}

Potevamo procedere anche come segue:
 
Calcoliamo il limite
\begin{align*}
\lim_{x\to 1^+}\frac{3x+2}{x-1}=&\\
\intertext{Dato che}
1^+-1>&0\\
3\cdot (1^+)+2>&0\\
\intertext{otteniamo}
\lim_{x\to 1^+}\frac{3x+2}{x-1}=&+\infty\\
\intertext{Calcoliamo il limite}
\lim_{x\to 1^-}\frac{3x+2}{x-1}=&\\
\intertext{Dato che}
1^--1<&0\\
3\cdot (1^-)+2>&0\\
\intertext{otteniamo}
\lim_{x\to 1^-}\frac{3x+2}{x-1}=&-\infty\\
\end{align*} 
\end{exercise}
\begin{exercise}[no solution]
	$f(x)= \dfrac{2x^2+x}{x^2+2x+1}$
\end{exercise}
\begin{exercise}
	Trovare gli asintoti richiesti
$y=\dfrac{x^2+4x+3}{x^2+3x+5}$
	\tcblower
$y=\dfrac{x^2+4x+3}{x^2+3x+5}$	
	
Determiniamo il dominio ponendo il denominatore uguale a zero
\begin{align*}
x^2+3x+5=&0\\ 
x_{1,2}=&\dfrac{-3\pm\sqrt{9-20}}{2}\\
\intertext{L'equazione non ha soluzioni reali}
\end{align*}
La funzione non ha asintoti verticali
\end{exercise}
\begin{exercise}Trovare gli asintoti richiesti
 $y=\dfrac{3x^2+1}{x^3+x^2}$
	\tcblower
 $y=\dfrac{3x^2+1}{x^3+x^2}$	
	
\begin{align*}
x^3+x^2=&0\\ 
x^2(x+1)=0\\
x_{1,2}=&\dfrac{0\pm\sqrt{0+0}}{2}\\
=&0\\ 
x+1=&0\\
x_3=&-1
\end{align*}
Quindi il dominio è $x\neq 0$ e $x\neq-1$

Studiamo  il segno della funzione.
\[y=\dfrac{3x^2+1}{x^3+x^2} \]
Dato che è una razionale fratta studiamo il segno del numeratore 
\begin{align*}
3x^2+1=&0\\ 
x_{1,2}=&\dfrac{0\pm\sqrt{0-12}}{2}\\
\intertext{L'equazione non ha soluzioni reali}
\end{align*}
Utilizzando i risultati precedenti otteniamo il grafico 
\begin{center}
	\includestandalone{quinto/grafici/distreparti}
\end{center}
In base a quanto detto 
\begin{align*}
\lim_{x\to 0}\dfrac{3x^2+1}{x^3+x^2}=&+\infty\\
\lim_{x\to 1^-}\dfrac{3x^2+1}{x^3+x^2}=&-\infty\\
\lim_{x\to 1^+}\dfrac{3x^2+1}{x^3+x^2}=&+\infty\\
\intertext{Otteniamo lo stesso risultato nel seguente modo:}
\lim_{x\to 0^+}\dfrac{3x^2+1}{x^3+x^2}=&\\
\intertext{Dato che}
3(0^+)^2+1>&0\\
x^3+x^2=&x^2(x+1)\\
(0^+)^2(0^++1)>&0
\intertext{otteniamo}
\lim_{x\to 0^+}\dfrac{3x^2+1}{x^3+x^2}=&+\infty
\intertext{continuiamo}
\lim_{x\to 0^-}\dfrac{3x^2+1}{x^3+x^2}=&
\intertext{Dato che}
3(0^-)^2+1>&0\\
x^3+x^2=&x^2(x+1)\\
(0^-)^2(0^-+1)>&0
\intertext{otteniamo}
\lim_{x\to 0^-}\dfrac{3x^2+1}{x^3+x^2}=&+\infty
\intertext{concludiamo}
\lim_{x\to -1^+}\dfrac{3x^2+1}{x^3+x^2}=&
\intertext{Dato che}
3(-1^+)^2+1>&0\\
x^3+x^2=&x^2(x+1)\\
(-1^+)^2(-1^++1)>&0
\intertext{otteniamo}
\lim_{x\to -1^+}\dfrac{3x^2+1}{x^3+x^2}=&+\infty
\intertext{continuiamo}
\lim_{x\to -1^-}\dfrac{3x^2+1}{x^3+x^2}=&
\intertext{Dato che}
3(-1^-)^++1>&0\\
x^3+x^2=&x^2(x+1)\\
(-1^-)^2(-1^-+1)<&0
\intertext{otteniamo}
\lim_{x\to -1^-}\dfrac{3x^2+1}{x^3+x^2}=&-\infty
\end{align*}
\end{exercise}
\begin{exercise}[no solution]
Trovare gli asintoti richiesti $f(x)= \dfrac{x+1}{2x-1}$
\end{exercise}
\section{Asintoti orizzontali}
Risolvi i seguenti esercizi
\begin{exercise}
	Trovare gli asintoti richiesti
 $y=\dfrac{3x^2+5x+2}{x^2-1}$
	\tcblower
	$y=\dfrac{3x^2+5x+2}{x^2-1}$
	
Una funzione ha un asintoto orizzontale  se esiste ed è finito il limite
\begin{equation*}
\lim_{x\to +\infty}\dfrac{3x^2+5x+2}{x^2-1}
\end{equation*}
questo limite è del tipo $\dfrac{\infty}{\infty}$, forma indeterminata. 
\begin{align*}
\lim_{x\to +\infty}\dfrac{3x^2+5x+2}{x^2-1}=&\\
\lim_{x\to +\infty}\dfrac{x^2(3+5\dfrac{x}{x^2}+\dfrac{2}{x^2})}{x^2(1-\dfrac{1}{x^2})}=&
\intertext{semplifico}
\lim_{x\to +\infty}\dfrac{(3+5\dfrac{x}{x^2}+\dfrac{2}{x^2})}{(1-\dfrac{1}{x^2})}=&
\intertext{i termini fra parentesi sono trascurabili, quindi}
\lim_{x\to +\infty}\dfrac{(3+5\dfrac{x}{x^2}+\dfrac{2}{x^2})}{(1-\dfrac{1}{x^2})}=3\\
\end{align*}
l'asintoto è $y=3$\index{Asintoto!orizzontale}
\end{exercise}
	\begin{exercise}[no solution]
Trovare gli asintoti richiesti $y=\dfrac{x^2+4x+3}{x^2+3x+5}$
\end{exercise}
\begin{exercise}
	Trovare gli asintoti richiesti
	$y=\dfrac{3x^2+1}{x^3+x^2}$
	\tcblower
	$y=\dfrac{3x^2+1}{x^3+x^2}$	
	
	\begin{align*}
	\lim_{x\to +\infty}\dfrac{3x^2+1}{x^3+x^2}=&\\
	\lim_{x\to +\infty}\dfrac{x^2(3+\dfrac{1}{x^2})}{x^3(1+\dfrac{1}{x})}=&
	\intertext{semplifico}
	\lim_{x\to +\infty}\dfrac{3+\dfrac{1}{x^2}}{x(1+\dfrac{1}{x})}=&
	\intertext{i termini fra parentesi sono trascurabili, quindi}
	\lim_{x\to +\infty}\dfrac{1(3+\dfrac{1}{x^2})}{x(1+\dfrac{1}{x})}=0\cdot 3=0\\
	\end{align*}
	L'asintoto è $y=0$, l'asse delle $x$.
\end{exercise}
\section{Asintoti obliqui}
Risolvere i seguenti esercizi
\begin{exercise}
	Trovare gli asintoti richiesti
	$y=\dfrac{2x^3+x^2+2}{x^2+1}$
	\tcblower
	$y=\dfrac{2x^3+x^2+2}{x^2+1}$	
	
	\begin{align*}
	m=&\lim_{x\to +\infty}\dfrac{2x^3+x^2+2}{x^2+1}\cdot\dfrac{1}{x}\\
	=&\lim_{x\to +\infty}\dfrac{2x^3+x^2+2}{x^3+x}\\
	=&\lim_{x\to +\infty}\dfrac{x^3\left(2+\dfrac{1}{x}+\dfrac{2}{x^3}\right)}{x^3\left(1+\dfrac{1}{x^2}\right)}\\
	\intertext{Dato che}
	\dfrac{x^3}{x^3}=&1\\
	\lim_{x\to +\infty}\dfrac{1}{x}=&0\\
	\lim_{x\to +\infty}\dfrac{2}{x^3}=&0\\
	\lim_{x\to +\infty}\dfrac{1}{x^2}=&0\\
	\intertext{otteniamo}
	m=&2\\
	q=&\lim_{x\to +\infty}\left[\dfrac{2x^3+x^2+2}{x^2+1}-2x\right]\\
	=&\lim_{x\to +\infty}\dfrac{2x^3+x^2+2-2x^3-2x}{x^2+1}\\
	=&\lim_{x\to +\infty}\dfrac{x^2-2x+2}{x^2+1}\\
	=&\lim_{x\to +\infty}\dfrac{x^2\left(1-\dfrac{2}{x}+\dfrac{2}{x^2}\right)}{x^2\left(1+\dfrac{1}{x^2}\right)}\\
	\intertext{Dato che}
	\dfrac{x^2}{x^2}=&1\\
	\lim_{x\to +\infty}\dfrac{1}{x}=&0\\
	\lim_{x\to +\infty}\dfrac{1}{x^2}=&0\\
	\intertext{otteniamo}
	q=&1\\
	\intertext{Quindi l'asintoto è}
	y=&2x+1
	\end{align*}\index{Asintoto!obliquo}
\end{exercise}
\begin{exercise}[no solution]
	Trovare gli asintoti richiesti $f(x)= \dfrac{2x^2+1}{x-1}$
\end{exercise}
\begin{exercise}
	Trovare gli asintoti richiesti
	$y=\dfrac{4x^4+1}{3x^3+1}$
	\tcblower
	$y=\dfrac{4x^4+1}{3x^3+1}$	
	
	\begin{align*}
	m=&\lim_{x\to +\infty}\dfrac{4x^4+1}{3x^3+1}\cdot\dfrac{1}{x}\\
	=&\lim_{x\to +\infty}\dfrac{4x^4+1}{3x^4+x}\\
	=&\lim_{x\to +\infty}\dfrac{x^4\left(4+\dfrac{1}{x^4}\right)}{x^4\left(3+\dfrac{1}{x^3}\right)}\\
	\intertext{Dato che}
	\dfrac{x^4}{x^4}=&1\\
	\lim_{x\to +\infty}\dfrac{1}{x^3}=&0\\
	\lim_{x\to +\infty}\dfrac{1}{x^4}=&0\\
	\intertext{otteniamo}
	m=&\dfrac{4}{3}\\
	q=&\lim_{x\to +\infty}\left[\dfrac{4x^4+1}{3x^3+1}-\dfrac{4}{3}x\right]\\
	=&\lim_{x\to +\infty}\dfrac{12x^4+3-12x^4-4x}{3x^3+1}\\
	=&\lim_{x\to +\infty}\dfrac{-4x+3}{9x^3+3}\\
	=&\lim_{x\to +\infty}\dfrac{x\left(-4+\dfrac{3}{x}\right)}{x^3\left(9+\dfrac{3}{x^3}\right)}\\
	\intertext{Dato che}
	\dfrac{x}{x^3}=&\dfrac{1}{x^2}\\
	\lim_{x\to +\infty}\dfrac{3}{x}=&0\\
	\lim_{x\to +\infty}\dfrac{1}{x^2}=&0\\
	\lim_{x\to +\infty}\dfrac{3}{x^2}=&0\\
	\intertext{otteniamo}
	q=&0\\
	\intertext{Quindi l'asintoto è}
	y=&\dfrac{4}{3}x
	\end{align*}\index{Asintoto!obliquo}
\end{exercise}
\section{Esercizi di riepilogo asintoti}
 
\begin{exercise}
	Trovare gli asintoti della funzione $f(x)=\dfrac{x^2+2x}{x-1}$
	\tcblower
	$f(x)=\dfrac{x^2+2x}{x-1}$
		
Iniziamo da quelli verticali. Determiniamo il dominio. Dato che è una razionale fratta il dominio si ottiene ponendo il denominatore uguale a zero\[x-1=0\] $x=1$. Il dominio è $\R-\lbrace1\rbrace$. Verifichiamo se $x=1$ è un asintoto verticale.
\begin{align*}
\lim_{x\to 1^+}\dfrac{x^2+2x}{x-1}=&\\
\intertext{Dato che}
(1^+)^2+2\cdot 1^+>&0\\
1^+-1>&0\\
\lim_{x\to 1^+}\dfrac{x^2+2x}{x-1}=&+\infty\\
\lim_{x\to 1^-}\dfrac{x^2+2x}{x-1}=&\\
\intertext{Dato che}
(1^-)^2+2\cdot 1^->&0\\
1^--1<&0\\
\lim_{x\to 1^-}\dfrac{x^2+2x}{x-1}=&-\infty\\
\end{align*}
Quindi $x=1$ è un asintoto verticale\index{Asintoto!verticale}
Dato che il grado del numeratore supera di uno il grado del denominatore verifichiamo se esiste un asintoto obliquo. Calcoliamo il limite
\begin{align*}
m=\lim_{x\to +\infty}\dfrac{x^2+2x}{x-1}\cdot\dfrac{1}{x}=&\\
=&\lim_{x\to +\infty}\dfrac{x^2+2x}{x^2-x}\\
=&\lim_{x\to +\infty}\dfrac{x^2\left(1+\dfrac{2}{x}\right)}{x^2\left(1-\dfrac{1}{x}\right)}\\
\intertext{Dato che}
\dfrac{x^2}{x^2}=&1\\
\lim_{x\to +\infty}\dfrac{1}{x}=&0\\
\intertext{otteniamo}
m=&1\\
q=&\lim_{x\to +\infty}\left[\dfrac{x^2+2x}{x-1}-x\right]\\
=&\lim_{x\to +\infty}\dfrac{x^2+2x-x^2-x}{x-1}\\
=&\lim_{x\to +\infty}\dfrac{x}{x-1}\\
=&\lim_{x\to +\infty}\dfrac{x}{x\left(1-\dfrac{1}{x}\right)}\\
\intertext{Dato che}
\dfrac{x}{x}=&1\\
\lim_{x\to +\infty}\dfrac{1}{x}=&0\\
\intertext{otteniamo}
q=&1\\
\end{align*}
L'asintoto cercato\[y=x+1\]
\end{exercise}
\begin{exercise}
	Trovare gli asintoti della funzione $f(x)=\dfrac{x^3+2}{x^3-1}$
	\tcblower
$f(x)=\dfrac{x^3+2}{x^3-1}$	
	
	Iniziamo da quelli verticali. Determiniamo il dominio. Dato che è una razionale fratta il dominio si ottiene ponendo il denominatore uguale a zero\[x^3-1=0\] $x=1$. Il dominio è $\R-\lbrace1\rbrace$. Verifichiamo se $x=1$ è un asintoto verticale.
	\begin{align*}
	\lim_{x\to 1^+}\dfrac{x^3+2}{x^3-1}=&\\
	\intertext{Dato che}
	(1^+)^3+2>&0\\
	(1^+)^3-1>&0\\
	\intertext{ottengo}
	\lim_{x\to 1^+}\dfrac{x^3+2}{x^3-1}=&+\infty\\
		\intertext{Calcoliamo}
	\lim_{x\to 1^-}\dfrac{x^3+2}{x^3-1}=&\\
		\intertext{Dato che}
	(1^-)^3+2>&0\\
	(1^-)^3-1<&0\\
	\intertext{ottengo}
	\lim_{x\to 1^-}\dfrac{x^2+2x}{x-1}=&-\infty\\
	\end{align*}
	Quindi $x=1$ è un asintoto verticale\index{Asintoto!verticale}
	Dato che il grado del numeratore e uguale al grado del denominatore verifichiamo se esiste un asintoto orizzontale. Calcoliamo il limite
	\begin{align*}
	\lim_{x\to +\infty}\dfrac{x^3+2}{x^3-1}=&\\
	\lim_{x\to +\infty}\dfrac{x^3\left(1+\dfrac{2}{x^3}\right)}{x^3\left(1-\dfrac{1}{x^3}\right)}=&\\
	\intertext{Dato che}
	\dfrac{x^3}{x^3}=&1\\
	\lim_{x\to +\infty}\dfrac{1}{x^3}=&0\\
	\intertext{otteniamo}
	\lim_{x\to +\infty}\dfrac{x^3+2}{x^3-1}=&1\\
	\end{align*}
	L'asintoto cercato\[y=1\]\index{Asintoto!orizzontale}
\end{exercise}
\begin{exercise}[no solution]
	Trovare gli asintoti della funzione
	$f(x)=\dfrac{1}{x+1}$
\end{exercise}
\begin{exercise}
	Trovare gli asintoti della funzione $f(x)=\dfrac{x^3+5x}{x+1}$
	\tcblower
		Trovare gli asintoti della funzione $f(x)=\dfrac{x^3+5x}{x+1}$
	
	Iniziamo da quelli verticali. Determiniamo il dominio. Dato che è una razionale fratta il dominio si ottiene ponendo il denominatore uguale a zero\[x+1=0\] $x=-1$. Il dominio è $\R-\lbrace-1\rbrace$. Verifichiamo se $x=-1$ è un asintoto verticale.
	\begin{align*}
	\lim_{x\to -1^+}\dfrac{x^3+5x}{x+1}=&\\
	\intertext{Dato che}
	(-1^+)^3+5\cdot(-1^+)<&0\\
	-1^+1>&0\\
	\intertext{ottengo}
	\lim_{x\to -1^+}\dfrac{x^3+5x}{x+1}=&-\infty\\
	\intertext{Calcoliamo}
	\lim_{x\to -1^-}\dfrac{x^3+5x}{x+1}=&\\
	\intertext{Dato che}
	(-1^-)^3+5\cdot(-1^-)<&0\\
-1^-1<&0\\
	\intertext{ottengo}
	\lim_{x\to -1^-}\dfrac{x^3+5x}{x+1}=&+\infty\\
	\end{align*}
	Quindi $x=1$ è un asintoto verticale\index{Asintoto!verticale}
	Verifichiamo se esiste un asintoto orizzontale. Calcoliamo il limite
	\begin{align*}
	\lim_{x\to +\infty}\dfrac{x^3+5x}{x+1}=&\\
	\lim_{x\to +\infty}\dfrac{x^3\left(1+\dfrac{5}{x^2}\right)}{x\left(1+\dfrac{1}{x}\right)}=&\\
	\intertext{Dato che}
	\dfrac{x^3}{x}=&x^2\\
	\lim_{x\to +\infty}\dfrac{5}{x^2}=&0\\
	\lim_{x\to +\infty}\dfrac{1}{x}=&0\\
	\intertext{otteniamo}
	\lim_{x\to +\infty}x^2=&+\infty\\
	\end{align*}
	L'asintoto non esiste\index{Asintoto!orizzontale}
	
	Verifichiamo che non ha asintoti obliqui. Calcoliamo il  limite:
	\begin{align*}
	m=&\lim_{x\to +\infty}\dfrac{x^3+5x}{x+1}\cdot\dfrac{1}{x}\\
	=&\lim_{x\to +\infty}\dfrac{x^3+5x}{x^2+x}\\
	=&\lim_{x\to +\infty}\dfrac{x^3\left(1+\dfrac{5}{x^2}\right)}{x^2\left(1+\dfrac{1}{x}\right)}\\
	\intertext{Dato che}
	\dfrac{x^3}{x^2}=&x\\
	\lim_{x\to +\infty}\dfrac{5}{x^2}=&0\\
	\lim_{x\to +\infty}\dfrac{1}{x}=&0\\
	\intertext{ottengo}
	=&\lim_{x\to +\infty}\dfrac{x^3+5x}{x^2+x}=+\infty
	\end{align*}
\end{exercise}
\begin{exercise}
Trovare gli asintoti della funzione $f(x)=\dfrac{x^2-2x+4}{3x-5}$
	\tcblower
$f(x)=\dfrac{x^2-2x+4}{3x-5}$

Iniziamo da quelli verticali. Determiniamo il dominio. Dato che è una razionale fratta il dominio si ottiene ponendo il denominatore uguale a zero\[3x-5=0\] $x=\dfrac{5}{3}$. Il dominio è $\R-\lbrace\dfrac{5}{3}\rbrace$. Verifichiamo se $x=\dfrac{5}{3}$ è un asintoto verticale. Studiamo il segno della funzione
\begin{align*}
\dfrac{x^2-2x+4}{3x-5}>&
x^2-2x+4=&0\\
\Delta=&b^2-4ac\\
=&4-16<&0\\
\intertext{Quindi il numeratore è sempre positivo}
3x-5>&0\\
x>&\dfrac{5}{3}\\
\intertext{Otteniamo il grafico}
\end{align*}
	\begin{center}
	\includestandalone{quinto/grafici/dissecdue}
\end{center}\index{Asintoto!verticale}
Quindi otteniamo i limiti
\end{exercise}

%\tcbstoprecording
%\newpage
%\section{Soluzione esercizi}
%\tcbinputrecords
