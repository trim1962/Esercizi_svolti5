\chapter{Asintoti}
\section{Asintoti verticali}
Risolvi i seguenti esercizi
\tcbstartrecording
%\begin{exercise}

%	\tcblower

%\end{exercise}
	%\begin{exercise}[no solution]
%	It holds:
%	\begin{equation*}
%	\frac{d}{dx}\left(\ln|x|\right) = \frac{1}{x}.
%	\end{equation*}
%\end{exercise}
\begin{exercise}
\[y=\dfrac{3x^2+5x+2}{x^2-1}\]
	\tcblower
Calcolo il dominio
\begin{align*}
x^2-1=&0\\ 
x_{1,2}=&\dfrac{0\pm\sqrt{0+4}}{2}\\
=&\dfrac{\pm 2}{2}\\
=&\begin{cases}
x_1=+1\\
x_2=-1
\end{cases}\\
\end{align*}
calcolo il limite 
\begin{equation*}
\lim_{x\to -1}\dfrac{3x^2+5x+2}{x^2-1}=\dfrac{0}{0}
\end{equation*}
Per risolvere questa forma indeterminata scompongo
il numeratore
\begin{align*}
3x^2+5x+2=&0\\
x_{1,2}=&\dfrac{-5\pm\sqrt{25-24}}{6}\\
=&\dfrac{-5\pm 1}{6}\\
=&\begin{cases}
x_1=-1\\
x_2=-\dfrac{2}{3}
\end{cases}\\
3x^2+5x+2=&3(x+\dfrac{2}{3})(x+1)
\end{align*}
quindi
\begin{align*}
\lim_{x\to -1}\dfrac{3x^2+5x+2}{x^2-1}=&\\
=&\lim_{x\to -1}\dfrac{3(x+\dfrac{2}{3})(x+1)}{(x-1)(x+1)}
\intertext{semplifico}
=&\lim_{x\to -1}\dfrac{3x+2}{x-1}\\
=&\dfrac{1}{2}
\end{align*}
Quindi $x=-1$ non è asintoto verticale. Utilizzando i calcoli precedenti calcolo il limite
\begin{equation*}
\lim_{x\to 1}\dfrac{3x^2+5x+2}{x^2-1}=\lim_{x\to 1}\dfrac{3x+2}{x-1}
\end{equation*}
Studio con una disequazione il segno della funzione
\begin{equation*}
y=\dfrac{3x+2}{x-1}\geq0
\end{equation*}
\begin{align*}
3x+2\geq&0\\
x\geq&-\dfrac{2}{3}\\
x-1>&0\\
x>&1
\end{align*}
ottengo il grafico
\begin{center}
	\includestandalone{quinto/grafici/disprimogradodueb}
\end{center}\index{Limite!infinito!tende finito}
Quindi la frazione è positiva per valori a destra di uno negativa a sinistra. Segue che
\begin{align*}
\lim_{x\to 1^+}\dfrac{3x+2}{x-1}=&+\infty\\
\lim_{x\to 1^-}\dfrac{3x+2}{x-1}=&-\infty\\
\end{align*}
Quindi $x=1$ è asintoto verticale
\end{exercise}
\tcbstoprecording
\newpage
\section{Soluzioni asintoti}
\tcbinputrecords
