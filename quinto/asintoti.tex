\chapter{Asintoti}
\section{Asintoti verticali}
Risolvi i seguenti esercizi
\tcbstartrecording
%\begin{exercise}

%	\tcblower

%\end{exercise}
	%\begin{exercise}[no solution]
%	It holds:
%	\begin{equation*}
%	\frac{d}{dx}\left(\ln|x|\right) = \frac{1}{x}.
%	\end{equation*}
%\end{exercise}
\begin{exercise}
\[y=\dfrac{x^2-5x+6}{x^2-1}\]
	\tcblower
Studio il dominio della funzione. 
\begin{align*}
x^2-1=&0\\ 
x_{1,2}=&\dfrac{0\pm\sqrt{0+4}}{2}\\
=&\dfrac{\pm 2}{2}\\
=&\begin{cases}
x_1=+1\\
x_2=-1
\end{cases}\\
\end{align*}
Quindi il dominio della funzione è  $x\neq\pm1$
Studio il segno della funzione
\begin{equation*}
\dfrac{x^2-5x+6}{x^2-1}\geq 0
\end{equation*}
Calcolo il limite 
\begin{equation*}\lim_{x\to -1}\frac{x^2-5x+6}{x^2-1}=\dfrac{1+6+6}{0}
\end{equation*}
Questo limite è un infinito
\end{exercise}
\begin{exercise}
\[y=\dfrac{3x^2+5x+2}{x^2-1}\]
	\tcblower
 il dominio
\begin{align*}
x^2-1=&0\\ 
x_{1,2}=&\dfrac{0\pm\sqrt{0+4}}{2}\\
=&\dfrac{\pm 2}{2}\\
=&\begin{cases}
x_1=+1\\
x_2=-1
\end{cases}\\
\end{align*}
Quindi il dominio della funzione è  $x\neq\pm1$

Calcolo il limite 
\begin{equation*}\lim_{x\to -1}\frac{3x^2+5x+2}{x^2-1}=\dfrac{0}{0}
\end{equation*}
Per risolvere questa forma indeterminata scompongo
il numeratore
\begin{align*}
3x^2+5x+2=&0\\
x_{1,2}=&\dfrac{-5\pm\sqrt{25-24}}{6}\\
=&\dfrac{-5\pm 1}{6}\\
=&\begin{cases}
x_1=-1\\
x_2=-\dfrac{2}{3}
\end{cases}\\
3x^2+5x+2=&3(x+\dfrac{2}{3})(x+1)
\end{align*}
quindi
\begin{align*}
\lim_{x\to -1}\dfrac{3x^2+5x+2}{x^2-1}=&\\
=&\lim_{x\to -1}\dfrac{3(x+\dfrac{2}{3})(x+1)}{(x-1)(x+1)}
\intertext{semplifico}
=&\lim_{x\to -1}\dfrac{3x+2}{x-1}\\
=&\dfrac{1}{2}
\end{align*}
Quindi $x=-1$ non è asintoto verticale. Utilizzando i calcoli precedenti ottengo il limite:
\begin{equation*}
\lim_{x\to 1}\dfrac{3x^2+5x+2}{x^2-1}=\lim_{x\to 1}\dfrac{3x+2}{x-1}
\end{equation*}
Studio con una disequazione il segno della funzione
\begin{equation*}
y=\dfrac{3x+2}{x-1}\geq0
\end{equation*}
\begin{align*}
3x+2\geq&0\\
x\geq&-\dfrac{2}{3}\\
x-1>&0\\
x>&1
\end{align*}
ottengo il grafico
\begin{center}
	\includestandalone{quinto/grafici/disprimogradodueb}
\end{center}\index{Limite!infinito!tende finito}
Quindi la frazione è positiva per valori a destra di uno negativa a sinistra. Segue che
\begin{align*}
\lim_{x\to 1^+}\dfrac{3x+2}{x-1}=&+\infty\\
\lim_{x\to 1^-}\dfrac{3x+2}{x-1}=&-\infty\\
\end{align*}
Quindi $x=1$ è asintoto verticale\index{Asintoto!verticale}
\end{exercise}
\begin{exercise}
\[y=\dfrac{x^2+4x+3}{x^2+3x+5}\]
	\tcblower
Determino il dominio ponendo il denominatore uguale a zero
\begin{align*}
x^2+3x+5=&0\\ 
x_{1,2}=&\dfrac{-3\pm\sqrt{9-20}}{2}\\
\intertext{L'equazione non ha soluzioni reali}
\end{align*}
La funzione non ha asintoti verticali
\end{exercise}
\begin{exercise}
\[y=\dfrac{3x^2+1}{x^3+x^2} \]
	\tcblower
\begin{align*}
x^3+x^2=&0\\ 
x^2(x+1)=0\\
x_{1,2}=&\dfrac{0\pm\sqrt{0+0}}{2}\\
=&0\\ 
x+1=&0\\
x_3=&-1
\end{align*}
Quindi il dominio è $x\neq 0$ e $x\neq-1$

Studio  il segno della funzione.
\[y=\dfrac{3x^2+1}{x^3+x^2} \]
Dato che è una razionale fratta studio il segno del numeratore 
\begin{align*}
3x^2+1=&0\\ 
x_{1,2}=&\dfrac{0\pm\sqrt{0-12}}{2}\\
\intertext{L'equazione non ha soluzioni reali}
\end{align*}
Utilizzando i risultati precedenti otteniamo il grafico 
\begin{center}
	\includestandalone{quinto/grafici/distreparti}
\end{center}
In base a quanto detto 
\begin{align}
\lim_{x\to 0}\dfrac{3x^2+1}{x^3+x^2}=&+\infty\\
\lim_{x\to 1^-}\dfrac{3x^2+1}{x^3+x^2}=&-\infty\\
\lim_{x\to 1^+}\dfrac{3x^2+1}{x^3+x^2}=&+\infty\\
\end{align}
\end{exercise}
\section{Asintoti orizzontali}
Risolvi i seguenti esercizi
\begin{exercise}
\[y=\dfrac{3x^2+5x+2}{x^2-1}\]
	\tcblower
Una funzione ha un asintoto orizzontale  se esiste ed è finito il limite
\begin{equation*}
\lim_{x\to +\infty}\dfrac{3x^2+5x+2}{x^2-1}
\end{equation*}
questo limite è del tipo $\dfrac{\infty}{\infty}$, forma indeterminata. 
\begin{align*}
\lim_{x\to +\infty}\dfrac{3x^2+5x+2}{x^2-1}=&\\
\lim_{x\to +\infty}\dfrac{x^2(3+5\dfrac{x}{x^2}+\dfrac{2}{x^2})}{x^2(1-\dfrac{1}{x^2})}=&
\intertext{semplifico}
\lim_{x\to +\infty}\dfrac{(3+5\dfrac{x}{x^2}+\dfrac{2}{x^2})}{(1-\dfrac{1}{x^2})}=&
\intertext{i termini fra parentesi sono trascurabili, quindi}
\lim_{x\to +\infty}\dfrac{(3+5\dfrac{x}{x^2}+\dfrac{2}{x^2})}{(1-\dfrac{1}{x^2})}=3\\
\end{align*}
l'asintoto è $y=3$\index{Asintoto!orizzontale}
\end{exercise}
	\begin{exercise}[no solution]
\[y=\dfrac{x^2+4x+3}{x^2+3x+5}\]
\end{exercise}
\begin{exercise}
	\[y=\dfrac{3x^2+1}{x^3+x^2} \]
	\tcblower
	\begin{align*}
	\lim_{x\to +\infty}\dfrac{3x^2+1}{x^3+x^2}=&\\
	\lim_{x\to +\infty}\dfrac{x^2(3+\dfrac{1}{x^2})}{x^3(1+\dfrac{1}{x})}=&
	\intertext{semplifico}
	\lim_{x\to +\infty}\dfrac{3+\dfrac{1}{x^2}}{x(1+\dfrac{1}{x})}=&
	\intertext{i termini fra parentesi sono trascurabili, quindi}
	\lim_{x\to +\infty}\dfrac{1(3+\dfrac{1}{x^2})}{x(1+\dfrac{1}{x})}=0\cdot 3=0\\
	\end{align*}
	L'asintoto è $y=0$, l'asse delle $x$.
\end{exercise}
\tcbstoprecording
\newpage
\section{Soluzioni asintoti}
\tcbinputrecords
