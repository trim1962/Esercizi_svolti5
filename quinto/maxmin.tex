\chapter{Massimi e minimi}
\section{Razionali intere}
Risolvi i seguenti esercizi
\tcbstartrecording
%\begin{exercise}

%	\tcblower

%\end{exercise}
%\begin{exercise}[no solution]
%	It holds:
%	\begin{equation*}
%	\frac{d}{dx}\left(\ln|x|\right) = \frac{1}{x}.
%	\end{equation*}
%\end{exercise}

 \begin{exercise}
 Calcolare i punti di massimo e di minimo della funzione $y=x \;  \left(x^{2} - 2 \right)^{2}$ 
 	\tcblower
 Calcolare i punti di massimo e di minimo della funzione $y=x \;  \left(x^{2} - 2 \right)^{2}$ 
 	\begin{align*}
 y=&x \;  \left(x - 2 \right)^{2}\\
 y'=&\OpD{x \;  \left(x - 2 \right)^{2}}\\
 =& \left(x - 2 \right)^{2} + 2 \; x \;  \left(x - 2 \right)\\
 \intertext{Raccogliendo il fattore comune}
 =& \left(x - 2 \right) \;  \left(3 \; x - 2 \right)
 \intertext{Otteniamo due disequazioni di primo grado}
 x-2\geq& 0\\
 x\geq&2\\
 3x-2\geq&0\\
 x\geq&\dfrac{2}{3}\\
 \end{align*}
 	Otteniamo il grafico
 \begin{center}
 	\includestandalone{quinto/grafici/disgraficoprob23}
 \end{center}
Per $x=\dfrac{2}{3}$ la derivata è positiva nulla negativa, quindi la funzione ha un massimo. Per $x=2$ la derivata è negativa nulla positiva, quindi la funzione ha un minimo. 
 \end{exercise}
 
\begin{exercise}
Calcolare i punti di massimo e di minimo della funzione $y=x^{2} \;  \left(x - 2 \right)^{2}$
	\tcblower
Calcolare i punti di massimo e di minimo della funzione $y=x^{2} \;  \left(x - 2 \right)^{2}$
	\begin{align*}
	y=&x^{2} \;  \left(x - 2 \right)^{2}\\
	y'=&\OpD{x^{2} \;  \left(x - 2 \right)^{2}}\\
	=& 2 \; x \;  \left(x - 2 \right)^{2} + 2 \; x^{2} \;  \left(x - 2 \right)\\
	\intertext{Raccogliendo il fattore comune}
	 =&4 \; x \;  \left(x - 2 \right) \;  \left(x - 1 \right)
\intertext{Otteniamo tre disequazioni di primo grado}
x\geq&0\\
x-2\geq& 0\\
x\geq&2\\
x-1\geq&0\\
x\geq&1\\
	\end{align*}
	Otteniamo il grafico
	\begin{center}
		\includestandalone{quinto/grafici/disgraficoprob24}
	\end{center}
Per $x=0$ la derivata è negativa nulla positiva, quindi la funzione ha un minimo. Per $x=1$ la derivata è positiva nulla negativa, quindi la funzione ha un massimo. Per $x=2$ la derivata è negativa nulla positiva, quindi la funzione ha un minimo.\index{Massimo}\index{Minimo} 
\end{exercise}
\begin{exercise}
	Calcolare i punti di massimo e di minimo della funzione $y=x^{3} \;  \left(x - 2 \right)^{2}$
		\tcblower
	Calcolare i punti di massimo e di minimo della funzione $y=x^{3} \;  \left(x - 2 \right)^{2}$
	\begin{align*}
	y=&x^{3} \;  \left(x - 2 \right)^{2}\\
	y'=&\OpD{x^{3} \;  \left(x - 2 \right)^{2}}\\
	=& 3 \; x^{2} \;  \left(x - 2 \right)^{2} + 2 \; x^{3} \;  \left(x - 2 \right)\\
	\intertext{Raccogliendo il fattore comune}
	=&x^{2} \;  \left(x - 2 \right) \;  \left(5 \; x - 6 \right)
	\intertext{Otteniamo tre disequazioni}
	x^2\geq&0\\
	x-2\geq& 0\\
	x\geq&2\\
5 \; x - 6\geq&0\\
	x\geq&\dfrac{6}{5}\\
	\end{align*}
	Otteniamo il grafico
	\begin{center}
		\includestandalone{quinto/grafici/disgraficoprob25}
	\end{center}
	Per $x=0$ la derivata è positiva nulla positiva, quindi la funzione ha un flesso\index{Flesso}. Per $x=\dfrac{6}{5}$ la derivata è positiva nulla negativa, quindi la funzione ha un massimo. Per $x=2$ la derivata è negativa nulla positiva, quindi la funzione ha un minimo.\index{Massimo}\index{Minimo} 
\end{exercise}
\begin{exercise}[no solution]
	Calcolare i punti di massimo e di minimo della funzione $y=x^{4} \;  \left(x - 2 \right)^{2}$
\end{exercise}
\begin{exercise}
Calcolare i punti di massimo e di minimo della funzione $y=6x^3+15x^2+12x+5$
	\tcblower
Calcolare i punti di massimo e di minimo della funzione $y=6x^3+15x^2+12x+5$
Calcolo il segno della derivata prima
\begin{align*}
y=&6x^3+15x^2+12x+5\\
y'=&\OpD{6x^3+15x^2+12x+5}\\
=&18 \; x^{2} + 30 \; x + 12\\
18 \; x^{2} + 30 \; x + 12\geq&0\\
18 \; x^{2} + 30 \; x + 12=&0\\
3 \; x^{2} + 5 \; x + 2=&0\\
x_{1,2}=&\dfrac{-5\pm\sqrt{25-24}}{6}\\
=&\dfrac{-5\pm 1}{6}\\
=&\begin{cases}
x_1=\dfrac{-5-1}{6}=-1\\
\\
x_2=\dfrac{-5+1}{6}=-\dfrac{2}{3}\\
\end{cases}
\end{align*}
	Otteniamo il grafico:
\begin{center}
	\includestandalone{quinto/grafici/disgraficoprob19}
\end{center}
Per $x=-1$ la derivata è positiva nulla negativa, quindi la funzione ha un massimo. Per $x=-\dfrac{2}{3}$ la derivata è negativa nulla positiva, quindi la funzione ha un minimo.\index{Massimo}\index{Minimo} 
\end{exercise}
\tcbstoprecording
\newpage
\section{Soluzione esercizi}
\tcbinputrecords
\newpage
\section{Razionali fratte}
\tcbstartrecording
\begin{exercise}
Calcolare i punti di massimo e di minimo della funzione $y=\dfrac{x^2+4}{x}$
	\tcblower
Calcolare i punti di massimo e di minimo della funzione $y=\dfrac{x^2+4}{x}$
Calcolo il segno della derivata prima
\begin{align*}
y=&\dfrac{x^2+4}{x}\\
y'=&\OpD{\dfrac{x^2+4}{x}}\\
=&\frac{2x^2-(x^2-4)}{x^{2}}\\
=&\frac{x^{2} - 4}{x^{2}}\\
=&\frac{x^{2} - 4}{x^{2}}\geq 0\\
\intertext{Ottengo due disequazioni}
x^2-4\geq 0\\
x_{1,2}=&\dfrac{\pm\sqrt{+16}}{2}\\
=&\dfrac{\pm 4}{2}\\
=&\begin{cases}
x_1=\dfrac{+4}{2}=+2\\
\\
x_2=\dfrac{-4}{2}=-2\\
\end{cases}
x^2>&0
\end{align*}
Ottengo il grafico
\begin{center}
	\includestandalone{quinto/grafici/disgraficoprob26}
\end{center}
Per $x=-2$ la derivata è positiva nulla negativa, quindi la funzione ha un massimo. Per $x=2$ la derivata è negativa nulla positiva, quindi la funzione ha un minimo. Per $x=0$ la  derivata e la funzione non esistono e la funzione non ha un flesso.\index{Massimo}\index{Minimo}\index{Flesso}
\end{exercise}
\begin{exercise}
Calcolare i punti di massimo e di minimo della funzione $y=\dfrac{x^{2} - 9}{x^{2} - 4}$
	\tcblower
Calcolare i punti di massimo e di minimo della funzione $y=\dfrac{x^{2} - 9}{x^{2} - 4}$
Calcolo il segno della derivata prima
\begin{align*}
y=&\dfrac{x^{2} - 9}{x^{2} - 4}\\
y'=&\OpD{\dfrac{x^{2} - 9}{x^{2} - 4}}\\
=&\frac{2 \; x \;  \left(x^{2} - 4 \right) - 2 \; x \;  \left(x^{2} - 9 \right)}{ \left(x^{2} - 4 \right)^{2}}\\
=&\frac{2x^3-8x-2x^3+18x}{(x^2-4)^2}\\
=&\frac{10x}{(x^2-4)^2}> 0\\
\intertext{Ottengo due disequazioni}
(x^2-4)^2>&0\\
\intertext{Il denominatore è sempre positivo e si annulla per}
x_{1,2}=&\dfrac{\pm\sqrt{+16}}{2}\\
=&\dfrac{\pm 4}{2}\\
=&\begin{cases}
x_1=\dfrac{+4}{2}=+2\\
\\
x_2=\dfrac{-4}{2}=-2\\
\end{cases}
\intertext{Il numeratore invece}
10 \;x\geq&0\\
x\geq&0\\
\end{align*}
Otteniamo il grafico
\begin{center}
	\includestandalone{quinto/grafici/disgraficoprob27}
\end{center}
 Per $x=0$ la derivata è negativa nulla positiva, quindi la funzione ha un minimo. Per $x=-2$ e per $x=+2$ la  derivata e la funzione non esistono e la funzione non ha un flesso.\index{Massimo}\index{Minimo}\index{Flesso}
\end{exercise}
\begin{exercise}
Calcolare i punti di massimo e di minimo della funzione $y=\dfrac{x^2+4x}{x^2+6x+5}$
	\tcblower
Calcolare i punti di massimo e di minimo della funzione $y=\dfrac{x^2+4x}{x^2+6x+5}$
Calcolo il segno della derivata prima
\begin{align*}
y=&\dfrac{x^2+4x}{x^2+6x+5}\\
y'=&\OpD{\dfrac{x^2+4x}{x^2+6x+5}}\\
=&\frac{ \left(2 \; x + 4 \right) \;  \left(x^{2} + 6 \; x + 5 \right) - \left(x^{2} + 4 \; x \right) \;  \left(2 \; x + 6 \right) }{ \left(x^{2} + 6 \; x + 5 \right)^{2}}\\
=&\frac{2x^3+12x^2+10x+4x^2+24x+20-2x^3-6x^2-8x^2-24x}{ \left(x^{2} + 6 \; x + 5 \right)^{2}}\\
=&\frac{2x^2+10x+20}{ \left(x^{2} + 6 \; x + 5 \right)^{2}}> 0\\
\intertext{Ottengo due disequazioni}
\left(x^{2} + 6 \; x + 5 \right)^{2}>&0\\
\intertext{Il denominatore è sempre positivo e si annulla per}
x_{1,2}=&\dfrac{-6\pm\sqrt{36-20}}{2}\\
=&\dfrac{-6\pm 4}{2}\\
=&\begin{cases}
x_1=\dfrac{-10}{2}=-5\\
\\
x_2=\dfrac{-2}{2}=-1\\
\end{cases}
\intertext{Il numeratore invece}
2x^2+10x+20\geq&0\\
x_{1,2}=&\dfrac{-10\pm\sqrt{100-160}}{2}\\
=&\dfrac{-10\pm\sqrt{-60}}{2}\\
\intertext{Delta negativo}
\end{align*}
Otteniamo il grafico
\begin{center}
	\includestandalone{quinto/grafici/disgraficoprob28}
\end{center}
 Per $x=-5$ e per $x=-1$ la  derivata e la funzione non esistono e la funzione non ha un flesso.  La derivata rimane di segno costante e la funzione non ha punti di minimo e di massimo\index{Massimo}\index{Minimo}\index{Flesso}
\end{exercise}
\tcbstoprecording
\newpage
\section{Soluzione esercizi}
\tcbinputrecords
\newpage