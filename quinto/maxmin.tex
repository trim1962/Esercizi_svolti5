\chapter{Massimi e minimi}
\section{Razionali intere}
Risolvi i seguenti esercizi
\tcbstartrecording
%\begin{exercise}

%	\tcblower

%\end{exercise}
%\begin{exercise}[no solution]
%	It holds:
%	\begin{equation*}
%	\frac{d}{dx}\left(\ln|x|\right) = \frac{1}{x}.
%	\end{equation*}
%\end{exercise}
\begin{exercise}
Calcolare i punti di massimo e di minimo della funzione $y=6x^3+15x^2+12x+5$
	\tcblower
Calcolare i punti di massimo e di minimo della funzione $y=6x^3+15x^2+12x+5$
Calcolo il segno della derivata prima
\begin{align*}
y=&6x^3+15x^2+12x+5\\
y'=&\OpD{6x^3+15x^2+12x+5}\\
=&18 \; x^{2} + 30 \; x + 12\\
18 \; x^{2} + 30 \; x + 12\geq&0\\
18 \; x^{2} + 30 \; x + 12=&0\\
3 \; x^{2} + 5 \; x + 2=&0\\
x_{1,2}=&\dfrac{-5\pm\sqrt{25-24}}{6}\\
=&\dfrac{-5\pm 1}{6}\\
=&\begin{cases}
x_1=\dfrac{-5-1}{6}=-1\\
\\
x_2=\dfrac{-5+1}{6}=-\dfrac{2}{3}\\
\end{cases}
\end{align*}
	Otteniamo il grafico:
\begin{center}
	\includestandalone{quinto/grafici/disgraficoprob19}
\end{center}
Per $x=-1$ la derivata è positiva nulla negativa, quindi la funzione ha un massimo. Per $x=-\dfrac{2}{3}$ la derivata è negativa nulla positiva, quindi la funzione ha un minimo. 
\end{exercise}
\tcbstoprecording
\newpage
\section{Soluzione esercizi}
\tcbinputrecords
\newpage
\section{Razionali fratte}
\tcbstartrecording
\newpage
\section{Soluzione esercizi}
\tcbinputrecords
\newpage